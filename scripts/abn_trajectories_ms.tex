% Options for packages loaded elsewhere
\PassOptionsToPackage{unicode}{hyperref}
\PassOptionsToPackage{hyphens}{url}
%
\documentclass[
]{article}
\usepackage{amsmath,amssymb}
\usepackage{lmodern}
\usepackage{ifxetex,ifluatex}
\ifnum 0\ifxetex 1\fi\ifluatex 1\fi=0 % if pdftex
  \usepackage[T1]{fontenc}
  \usepackage[utf8]{inputenc}
  \usepackage{textcomp} % provide euro and other symbols
\else % if luatex or xetex
  \usepackage{unicode-math}
  \defaultfontfeatures{Scale=MatchLowercase}
  \defaultfontfeatures[\rmfamily]{Ligatures=TeX,Scale=1}
\fi
% Use upquote if available, for straight quotes in verbatim environments
\IfFileExists{upquote.sty}{\usepackage{upquote}}{}
\IfFileExists{microtype.sty}{% use microtype if available
  \usepackage[]{microtype}
  \UseMicrotypeSet[protrusion]{basicmath} % disable protrusion for tt fonts
}{}
\makeatletter
\@ifundefined{KOMAClassName}{% if non-KOMA class
  \IfFileExists{parskip.sty}{%
    \usepackage{parskip}
  }{% else
    \setlength{\parindent}{0pt}
    \setlength{\parskip}{6pt plus 2pt minus 1pt}}
}{% if KOMA class
  \KOMAoptions{parskip=half}}
\makeatother
\usepackage{xcolor}
\IfFileExists{xurl.sty}{\usepackage{xurl}}{} % add URL line breaks if available
\IfFileExists{bookmark.sty}{\usepackage{bookmark}}{\usepackage{hyperref}}
\hypersetup{
  pdftitle={The impermanence of cropland abandonment limits potential for environmental benefits (draft)},
  pdfauthor={Christopher L. Crawford,\^{}a\^{}* He Yin,\^{}b Volker Radeloff,\^{}c and David S. Wilcove\^{}\{a, d\}},
  hidelinks,
  pdfcreator={LaTeX via pandoc}}
\urlstyle{same} % disable monospaced font for URLs
\usepackage[margin=1in]{geometry}
\usepackage{longtable,booktabs,array}
\usepackage{calc} % for calculating minipage widths
% Correct order of tables after \paragraph or \subparagraph
\usepackage{etoolbox}
\makeatletter
\patchcmd\longtable{\par}{\if@noskipsec\mbox{}\fi\par}{}{}
\makeatother
% Allow footnotes in longtable head/foot
\IfFileExists{footnotehyper.sty}{\usepackage{footnotehyper}}{\usepackage{footnote}}
\makesavenoteenv{longtable}
\usepackage{graphicx}
\makeatletter
\def\maxwidth{\ifdim\Gin@nat@width>\linewidth\linewidth\else\Gin@nat@width\fi}
\def\maxheight{\ifdim\Gin@nat@height>\textheight\textheight\else\Gin@nat@height\fi}
\makeatother
% Scale images if necessary, so that they will not overflow the page
% margins by default, and it is still possible to overwrite the defaults
% using explicit options in \includegraphics[width, height, ...]{}
\setkeys{Gin}{width=\maxwidth,height=\maxheight,keepaspectratio}
% Set default figure placement to htbp
\makeatletter
\def\fps@figure{htbp}
\makeatother
\setlength{\emergencystretch}{3em} % prevent overfull lines
\providecommand{\tightlist}{%
  \setlength{\itemsep}{0pt}\setlength{\parskip}{0pt}}
\setcounter{secnumdepth}{5}
\newcommand{\beginsupplement}{\setcounter{figure}{0} \setcounter{section}{0} \setcounter{table}{0} \setcounter{equation}{0} \renewcommand{\thefigure}{S\arabic{figure}} \renewcommand{\thesection}{S\arabic{section}} \renewcommand{\thetable}{S\arabic{table}} \renewcommand\theequation{S\arabic{equation}}}
\ifluatex
  \usepackage{selnolig}  % disable illegal ligatures
\fi
\newlength{\cslhangindent}
\setlength{\cslhangindent}{1.5em}
\newlength{\csllabelwidth}
\setlength{\csllabelwidth}{3em}
\newenvironment{CSLReferences}[2] % #1 hanging-ident, #2 entry spacing
 {% don't indent paragraphs
  \setlength{\parindent}{0pt}
  % turn on hanging indent if param 1 is 1
  \ifodd #1 \everypar{\setlength{\hangindent}{\cslhangindent}}\ignorespaces\fi
  % set entry spacing
  \ifnum #2 > 0
  \setlength{\parskip}{#2\baselineskip}
  \fi
 }%
 {}
\usepackage{calc}
\newcommand{\CSLBlock}[1]{#1\hfill\break}
\newcommand{\CSLLeftMargin}[1]{\parbox[t]{\csllabelwidth}{#1}}
\newcommand{\CSLRightInline}[1]{\parbox[t]{\linewidth - \csllabelwidth}{#1}\break}
\newcommand{\CSLIndent}[1]{\hspace{\cslhangindent}#1}

\title{The impermanence of cropland abandonment limits potential for environmental benefits (draft)}
\author{Christopher L. Crawford,\(^a\)\(^*\) He Yin,\(^b\) Volker Radeloff,\(^c\) and David S. Wilcove\(^{a, d}\)}
\date{May 07, 2021}

\begin{document}
\maketitle
\begin{abstract}
\(^a\)Princeton School of Public and International Affairs, Princeton University, Princeton, NJ\\
\(^b\)Department of Geography, Kent State University, Kent, OH\\
\(^c\)Department of Forest \& Wildlife Ecology, University of Wisconsin - Madison, Madison, WI\\
\(^d\)Department of Ecology \& Evolutionary Biology, Princeton University, Princeton, NJ\\
~\\
\(^*\)Corresponding Author, \href{mailto:ccrawford@princeton.edu}{\nolinkurl{ccrawford@princeton.edu}}, Robertson Hall, Princeton University, Princeton, NJ
\end{abstract}

{
\setcounter{tocdepth}{2}
\tableofcontents
}
\hypertarget{abstract}{%
\section{Abstract}\label{abstract}}

\emph{Must be \textless250 words (currently 238)}

Agricultural expansion is a major cause of land-use change globally, but millions of hectares of cropland are being abandoned as a result of demographic, economic, and environmental changes.
These abandoned croplands could be immensely valuable for carbon sequestration and biodiversity restoration.
However, their environmental value depends on the duration and persistence of abandonment, which is poorly known.
Here, we use high-resolution satellite imagery to examine the timing and duration of abandonment at eleven sites across four continents that experienced cropland abandonment from 1987 and 2017.
We find that abandonment is by and large fleeting, lasting on average only 14.42 years (SD = 7.78).\footnote{\emph{Note: both values reported are the means of each site's corresponding summary statistics, i.e.~the mean of means and mean of standard deviations}}
At most sites, more than 50\% of abandoned croplands were projected to be recultivated within 30 years, and some sites, particularly in Eastern Europe, Russia, and the US, showed accelerating rates of recultivation.
If current rates of abandonment continue, the mean duration of abandonment at most sites will likely plateau between 10 and 22.3 years by about 2040.
Counter to optimistic assumptions, most abandonment is unlikely to be permanent and will produce little in the way of benefits for carbon or biodiversity.
New policies and incentives will be needed to lengthen the period of abandonment so as to generate these and other benefits.
Until then, abandoned croplands will remain untapped opportunities.

\hypertarget{significance-statement}{%
\section{Significance Statement}\label{significance-statement}}

\emph{Must be \textless120 words (currently 121)}

Demographic, economic, and environmental changes often result in cropland abandonment.
While sometimes seen as a threat to food security and cultural landscapes, abandonment is increasing billed as an opportunity for habitat restoration and carbon sequestration.
However, those environmental benefits largely depend on how long ecosystems can regenerate in abandoned croplands and accumulate carbon and biodiversity, which takes decades in many ecosystems.
Existing studies often rely on single snapshots in time to estimate abandonment, and ignored recultivation.
Using annual land-cover maps, we track abandonment and recultivation at eleven sites across four continents that have recently experienced abandonment.
Contrary to previous assumptions, abandonment is short-lived and lands are frequently recultivated.
Until policymakers take steps to reduce recultivation, abandonment will remain a missed opportunity.

\hypertarget{keywords}{%
\subsection{Keywords}\label{keywords}}

\hypertarget{introduction}{%
\section{Introduction}\label{introduction}}

Populations are in flux around the world, as people seek new economic opportunities in cities and flee changing environments and conflicts.
Coupled with environmental changes and other factors that render some agricultural lands economically inviable, this rural outmigration has contributed to a growing global trend of agricultural abandonment.
In a world with increasing competition for land, abandoned agricultural lands are highly sought after for diverse goals such as increased cultivation, biofuel production, and carbon sequestration; to conservation scientists, this land represents a potentially large source of new habitat for wildlife as vegetation regenerates.

However, understanding agricultural abandonment and its potential impacts on biodiversity requires not only detailed information on where and how abandonment is taking place, but critically, what happens to abandoned lands after they are abandoned.
Most estimates of abandonment are either inferred from aggregated estimates of regional cultivation (e.g.~from FAO data on cultivated area; (Isbell et al. 2019; Munroe et al. 2013)), or from estimates of the gross area abandoned in a given year (Lark, Salmon, and Gibbs 2015).
Both approaches lack spatial and temporal detail on the trajectories taken by individual pieces of land, which is critical for understanding the environmental implications of abandonment (Yin et al. 2020).

Recent efforts to quantify the potential for biofuel production in abandoned agricultural lands present a good example.
Most of these studies use two snapshots in time, estimating abandoned lands as those in agriculture in the early time step and non-agriculture in the later time step.
For example, Næss, Cavalett, and Cherubini (2021) use land cover maps from the ESA-CCI from 1992 and 2015, identifying 83 million hectares of abandoned cropland globally.
They also review a range of similar estimates of abandonment that differ in geography and time scale, but all share a similar challenge: estimates of abandonment at a particular instant in time may represent significant overestimates as a result of dynamic patterns of abandonment and recultivation.
Key papers they cite are: Ramankutty and Foley (1999) (207 Mha), Campbell et al. (2008) (286 Mha), Estel et al. (2015), Li et al. (2018), Alcantara et al. (2013), Yu and Lu (2018), Yu et al. (2019).

Recent advances in satellite imagery-based mapping have made it possible to produce maps of agricultural abandonment with both high spatial resolution and temporal resolution, allowing us to ask questions about the timing and trajectories of agricultural abandonment.
Here, we utilize a new time-series of agricultural abandonment derived from publicly available Landsat imagery (1987-2017, 31 years) to investigate how long abandonment lasts, how abandoned lands ``decay,'' and which factors seem to determine the length of abandonment and recultivation.
Specifically, we ask the following questions:

\begin{enumerate}
\def\labelenumi{\arabic{enumi}.}
\tightlist
\item
  How long is abandoned land typically abandoned for, and how does this differ across sites?
\item
  What do abandonment decay rates (recultivation rates) look like at each site, and how do these differ between sites and through time?
\item
  Finally, what factors are most important for determining which pieces of land remain abandoned for longer periods of time, and which lands are recultivated?
\end{enumerate}

We use case studies of Brazil (2), the US (2), Southern \& Eastern Europe (4), the Middle East (1), and China (2) to provide a better understanding of how these patterns vary across regions (Yin et al. 2020) (see Figure \ref{fig:site-locations}).

Yin et al. (2020) established several methodological advances, including improvement in the mapping of 1) the timing of initial abandonment (year first abandoned), and 2) the abandonment rate (amount of abandoned land in a given year divided by the amount of agricultural land in either a) the previous year or b) the beginning of the time series). However, Yin et al.~did not assess the persistence of abandonment (how long abandoned lands stay abandoned for), frequency and patterns of recultivation, spatial predictors of abandonment persistence and recultivation, patterns of fragmentation in the resulting landscape, nor the implications of these for biodiversity. (These last two are likely for a second paper.)

Previous studies of Costa Rica (Reid et al. 2018) and the Amazon Smith et al. (2020) have focused on the persistence and timing of secondary forest growth and age (as identified by the presence of woody vegetation regrowth), but not exclusively on abandoned agricultural lands.
Identifying forest regeneration by picking up the signal of woody regrowth in satellite imagery may miss abandonment at earlier stages or that is regenerating into non-forest habitats (Yin et al. 2020).
Focusing on abandonment itself, not only to detecting the regeneration of woody vegetation, provides a better understanding of the outcomes of abandonment, and provides critical context for understanding the potential that abandonment holds for habitat regeneration, carbon sequestration, and other land uses.

Many studies of regeneration also do not consider patterns of persistence, or age of abandonment or regeneration (Yin et al. 2020).
Crouzeilles et al. (2020) only consider pixels as regenerated if they were in forest in 2016 and had been forest for at least 3 years, without assessing average persistence.
Aide et al. (2019) assess changes in woody vegetation over a 14 year period, but do so by averaging across years, losing the temporal pattern or regeneration and recultivation.
Yin et al. (2020) define land as abandoned when it has been continuously uncultivated for 5 years, but do not assess the length of time abandoned in detail.

\emph{Add additional detail on what makes our study special.}
This will be the first study with multiple sites and a global perspective. It also:
- makes use of more accurate and finer resolution abandonment maps
- covers a longer time period than other recent studies
- provides more detail on not just when recultivation happens, but the total time abandoned land is actually abandoned for
- looks at predictors of abandonment length, not just whether abandonment happens at all.
- characterizes this recultivation as decay rates for the first time, and looks at trends in this by site, and over time.

These simple insights about the trajectories taken by abandoned agricultural land will provide much needed context to broad statements about recent and future abandonment and the contribution that abandonment many make to habitat regeneration and biodiversity conservation (Chazdon et al. 2020).
The bulk of abandoned land may be relatively short-lived, from the perspective of ecological succession or carbon sequestration, and unlikely to regenerate into valuable habitat on its own without active policies to protect this land and encourage regeneration.

\hypertarget{research-questions}{%
\subsection{Research questions}\label{research-questions}}

Our primary goal is to understand the temporal nature of agricultural abandonment by exploring the trajectories taken by each piece of abandoned land through time.
Our main questions are simple: How long is abandoned farmland really ``abandoned?'' If some farmland is not truly abandoned, what happens to it?
Abandonment is not a permanent phenomenon.
In fact, the period of abandonment varies quite a bit around the world and through time.
In order to assess the effect of abandonment on biodiversity, carbon sequestration, and food security, we need a better understanding of how long agricultural abandonment lasts, how abandoned lands ``decay,'' and which factors seem to predict the timing and decay.

In pursuit of this understanding, we address two specific questions:

\begin{enumerate}
\def\labelenumi{\arabic{enumi}.}
\tightlist
\item
  How long is land actually abandoned for, on average, and how does abandonment length vary through space (i.e.~at different sites)?
\item
  How quickly is land recultivated, as measured by ``abandonment decay rates?''

  \begin{enumerate}
  \def\labelenumii{\alph{enumii}.}
  \tightlist
  \item
    How do these decay rates vary through time at each site, and how do they vary across sites? (What proportion of abandoned land remains abandoned long-term {[}e.g.~\textgreater20 years{]})
  \end{enumerate}
\end{enumerate}

Together, these questions provide a clearer picture of the timing and persistence of abandonment at various sites around the world, improving our understanding the conservation implications of agricultural abandonment, and providing crucial context to inform policies designed to manage abandoned lands.

\hypertarget{results}{%
\section{Results}\label{results}}

We find substantial amounts of cropland abandonment at each of our sites, ranging between 313134 ha (Shaanxi) and 924711 (Orenburg) ha as of 2017, for a total of 5.71 Mha of abandoned croplands (12.46\% of 45.8 Mha total land area across sites).
Our sites varied in size, and abandonment accounting for between 7.91\% (Nebraska) and 19.91\% (Shaanxi) of the total area of each site.
The notable exception was Mato Grosso, which only saw 9002 ha of abandoned cropland as of 2017 (0.27\% of total site area).
Across all sites, we observe a total of XXX ha of land abandoned as of 2017, across a site footprint of ha in total.
The area in each land cover class is shown for each site in Figure \ref{fig:area-by-lc}, and the area abandoned is shared by age class in Figure \ref{fig:area-abn-by-age-class}.
Total area abandoned is shown in Figure \ref{fig:area-abn-panel}. \emph{Include more description of the point of each of the figures, rather than just what they show.}

However, we also find that a significant area of these abandoned croplands were recultivated by the end of our time series (on average, 37.77\% of abandoned area across all sites; Figure \ref{fig:recult-by-threshold}).
As a result of this recultivation, the mean duration of abandonment across all sites was short: 14.42 years (SD = 7.78), with a minimum site mean duration of 12.86 years (Orenburg) and a maximum of 17.7 years (Bosnia \& Herzegovina; see Figure \ref{fig:mean-abn-length}).



\begin{figure}
\includegraphics[width=1\textwidth]{/Users/christophercrawford/Google Drive/_Projects/abandonment_trajectories/output/plots/_2021_03_13/mean_lengths_2021_03_13} \caption{Mean length of time abandoned (years) throughout our time series across study sites. The mean abandonment length is calculated for both the longest period of abandonment for each pixel throughout the time series (maximum length; in blue), and for all periods of abandonment throughout the time series (a single pixel may go through multiple periods of abandonment through; in red).}\label{fig:mean-abn-length}
\end{figure}

\hypertarget{land-cover-outcomes-of-abandonment}{%
\subsection{Land cover outcomes of abandonment}\label{land-cover-outcomes-of-abandonment}}

Abandoned cropland followed different successional trajectories at each site, where the proportion of abandoned cropland distributed between forest and grassland (as of 2017) ranging from a minimum of 1\% in forest (99 in grassland) at Orenburg to a maximum of 60\% in forest (40 in grassland) at Wisconsin (see Figure \ref{fig:abn-prop-lc}).



\begin{figure}
\includegraphics[width=1\textwidth]{/Users/christophercrawford/Google Drive/_Projects/abandonment_trajectories/output/plots/_2021_03_13/ms_panel_2021_03_13_s} \caption{Abandonment at a single site (Shaanxi/Shanxi Provinces, China), showing a) accumulation of abandoned land by age class, b) decay of abandoned land by year abandoned, c) the area in each land cover class through time (including both land that has been abandoned for five or more years, as well as any land abandoned for 1 or more years, therefore including short-term fallows), and d) the annual turnover of abandoned land through time.}\label{fig:abn-panel-s}
\end{figure}

\hypertarget{decay-abandonment-persistence-and-recultivation}{%
\subsection{Decay: abandonment persistence and recultivation}\label{decay-abandonment-persistence-and-recultivation}}

We tracked groups (``cohorts'') of cropland that were abandoned in the same year through time, modeling the proportion remaining abandoned (i.e.~recultivated) as a function of time elapsed since initial abandonment (Figure \ref{fig:abn-decay}).
By doing so, we are able to predict how long a given abandoned cropland will persist for before it is recultivated, regardless of where it took place during the time series.

\begin{itemize}
\tightlist
\item
  Site mean decay rates are shown in Figure X.
\item
  Some of the sites show accelerating recultivation (BH), while others show recultivation slowing down (Shaanxi), but coefficients are largely not significantly different from zero.
\end{itemize}

Result in text: interpreting the decay plots to understand, on average, what proportion of abandoned land remains abandoned long-term (e.g.~\textgreater20 years): this is simply looking at the 20 year mark, and noting the proportion remaining at that point. (on average, this ranges between about 67\% for Shaanxi and 5\% for Orenburg. Most sites cluster around 50\%.)





\begin{figure}
\includegraphics[width=1\textwidth]{/Users/christophercrawford/Google Drive/_Projects/abandonment_trajectories/output/plots/_2021_03_13/decay/model_decay_curves_by_site_l3_wide_2021_03_13} \caption{Mean decay trajectories for each site, based on the mean coefficients across all cohorts of abandoned land.}\label{fig:decay-curves-by-site}
\end{figure}



\begin{figure}
\includegraphics[width=1\textwidth]{/Users/christophercrawford/Google Drive/_Projects/abandonment_trajectories/output/plots/_2021_03_13/decay/l3_cohort_coefs_rate_of_change_v_2021_03_13} \caption{The rate of change of decay rates through time at each site. Values correspond to the change in the time required for a cohort of abandoned land to decline by half, in terms of years per year.}\label{fig:decay-rate-of-change}
\end{figure}

\hypertarget{extrapolation}{%
\subsection{Extrapolation}\label{extrapolation}}

Based on our modeled decay rates, we can extrapolate into the future by making some simple assumptions, and asking: if abandonment and decay trends stay about the same, how much abandoned land will accumulate, and how old will it be on average?

We make two assumptions in these extrapolations:\\
1. That a constant amount of land is newly abandoned each year at each site (based on the average amount of land abandoned each year - these individual amounts are shown as the ``Area Gained'' in panel d of Figures \ref{fig:panel-b}-\ref{fig:panel-w}).
2. That all abandoned lands at a given site decay at the same rate, as determined by the mean decay rate for that site (based on the mean of the \texttt{lm} coefficients across all years of abandonment).

These results are shown in Figure \ref{fig:extrapolation-combo}, along with a more detailed breakdown in Figure \ref{fig:extrapolation-area-by-age}, which shows the age of abandoned lands by age class.



\begin{figure}
\includegraphics[width=1\textwidth]{/Users/christophercrawford/Google Drive/_Projects/abandonment_trajectories/output/plots/_2021_03_13/decay/extrapolation_combo_tall_2021_03_13} \caption{The results of our simple extrapolation, including a) the mean decay trend for each site also shown in Figures \ref{fig:decay-curves-by-site}-\ref{fig:decay-time-to-range}, b) the mean age of abandonment, and c) the total area abandoned into the future. Colors corresponding to each site are consistent across the three panels.}\label{fig:extrapolation-combo}
\end{figure}







\begin{figure}
\includegraphics[width=1\textwidth]{/Users/christophercrawford/Google Drive/_Projects/abandonment_trajectories/output/plots/_2021_03_13/decay/extrapolate_age_by_age_bin_2021_03_13} \caption{Extrapolating area by age bin.}\label{fig:extrapolation-area-by-age}
\end{figure}

We considered a range of alternatives to these two assumptions, as detailed in Section \ref{section-extrapolation-si}, but found these general pattern to hold. If anything, our initial assumptions yield a more optimistic picture of how average duration of abandonment will change into the future.

\hypertarget{discussion}{%
\section{Discussion}\label{discussion}}

Area abandoned is likely to be higher in our sites than across the globe in general, given that we explicitly chose sites that have seen large amounts of recent abandonment.

\begin{itemize}
\tightlist
\item
  If decay rates remain about the same at each site (site mean), and a similar (and constant) area of cropland is newly abandoned each year, our simple extrapolation indicates that the impermanence we observe is not merely an artifact of the narrow window of our time series.
  Even the site with the most durable abandonment is unlikely to see a large amount of land persist for longer than XX years.
  In order to increase the environmental benefits provided
  Policy makers could address for increased attention policy
\end{itemize}

The definition of abandonment makes a big difference on both the mean abandonment duration (Figure \ref{fig:abn-thresholds-mean-duration}), and the recultivation rate (\ref{fig:abn-thresholds-recultivation}).
It is possible that in many sites, a longer definition of ``abandonment'' would be more appropriate.
Belarus \& Herzegoniva displays a strong bifurcated pattern, where some abandonment remains relatively durable and long-lasting, while other abandonment is recultivated relatively quickly.

Natural regeneration can represent a viable ecosystem restoration strategy, when given enough time (Zivec et al.~2021, others)
Consequences for water provision (Khorchani et al.~2021, others)

\hypertarget{materials-and-methods}{%
\section{Materials and Methods}\label{materials-and-methods}}

\hypertarget{definitions-of-abandonment}{%
\subsection{Definitions of abandonment}\label{definitions-of-abandonment}}

Our focus relates to how we define ``abandonment,'' specifically showing that in many places around the world, abandonment is not a permanent phenomenon.
Differentiating ``true'' abandonment from less permanent land-use changes, such as short-term fallowing or crop rotations, is a challenge because definitions can vary widely by region and by study.
In fact, abandonment is best viewed as a land-use transition in which land may pass through a spectrum of stages (Munroe et al. 2013).
Here, when we refer to ``abandonment,'' we refer to agricultural land that is no longer under active cultivation and is left free of direct human influence.
We do not refer to changes from intensive cropped to less intensive uses, such as extensive grazing.
In order to exclude short-term fallow periods, we define land as abandoned when it remains uncultivated for five or more consecutive years {[}following FAO{]}.
We assess the influence of this threshold in SI Section \ref{abn-thresholds}.
We focus exclusively on cropland abandonment, excluding the abandonment of pastures, which is more difficult to discern from satellite imagery (though recent advances have been made; see citations) and is not captured into our data.

\hypertarget{abandonment-maps}{%
\subsection{Abandonment maps}\label{abandonment-maps}}

Our analysis builds on maps developed in Yin et al. (2020). Yin et al. (2020) used the following approach: {[}\emph{insert short summary of Yin et al. (2020)}{]}.
We selected 11 sites from Yin et al. (2020) that were mapped with high accuracy {[}\emph{ask He for stat to insert here}{]} and provided broad coverage of different continents and ecosystems (see Figure \ref{fig:site-locations}).
These sites were chosen by Yin et al. (2020) as places that were highly likely to have experienced agricultural abandonment since the mid-1980s, for a variety of socioeconomic, political, and environmental reasons.
Due to this selection, our results are likely optimistic about the overall prevalence of abandoned agricultural lands in all agricultural regions, but instead more accurately represent areas that have recently experienced abandonment.



\begin{figure}
\includegraphics[width=1\textwidth]{/Users/christophercrawford/Google Drive/_Projects/abandonment_trajectories/output/plots/site_locations_w_labels_long} \caption{Sites included in this study, from Yin et al. (2020).}\label{fig:site-locations}
\end{figure}



\begin{figure}
\includegraphics[width=1\textwidth]{/Users/christophercrawford/Google Drive/_Projects/abandonment_trajectories/output/plots/_2021_03_13/site_locations_w_abn_age_gray_2021_03_13} \caption{Site locations with observed abandonment duration (in years) as of 2017.}\label{fig:site-locations-w-age}
\end{figure}

\hypertarget{data-processing-and-filtering}{%
\subsection{Data processing and filtering}\label{data-processing-and-filtering}}

We processed and analyzed our abandonment map data in RStudio version 1.4.993, using R version 4.0.2 (2020-06-22), relying heavily on the \texttt{raster} (Hijmans 2020) and \texttt{data.table} packages (Dowle and Srinivasan 2021).
Resource-intensive data processing was conducted on Princeton's research computing clusters.

These land cover maps contained four classes: 1) cropland, 2) herbaceous vegetation (e.g.~grassland), 3) woody vegetation (e.g.~forests), and 4) non-vegetation (e.g.~water, urban, or barren land), mapped annually from 1987 through 2017.
We identified periods of cropland abandonment by tracking each pixel's land cover classification through time and looking for land-cover changes that indicated transitions between agricultural activity and abandonment.
Taking a conservative approach, we considered agricultural activity to include both stable cultivation (continuous classification as cropland) and cyclical fallowing (multiple years of cropland followed by multiple years of non-cropland that do not meet our abandonment threshold).
We classified a pixel as ``abandoned'' when it transitioned from cropland to either herbaceous or woody vegetation (collectively referred to as ``non-cropland''), and subsequently remained classified as non-cropland for five or more consecutive years (following the FAO's definition of abandonment).
This transition from cropland to abandonment could take place at any point during the time-series, allowing for the abandonment of long-term agricultural lands as well as newly converted lands.
We considered a pixel to be ``recultivated'' if it transitioned from abandoned back to cropland (i.e.~when five or more years of continuous non-cropland were followed by cropland), therefore marking the end of that period of abandonment.

Pixels that remained in cropland or non-cropland classes throughout the entire time series were excluded, as were periods of non-cropland that began in the first year of the time series, even if that pixel was later classified as cropland and subsequently abandoned (leaving only periods of abandonment that we could verify had followed agricultural activity during our time series).
Pixels that transitioned from cropland to the non-vegetation class were not considered ``abandoned,'' and therefore we excluded all non-vegetation pixels from our analysis. \emph{\footnote{These pixels represent a very small portion of our dataset, and at most sites non-vegetated land remained about constant or declined over time.}}

In order to address potential classification errors in a single year, we implemented a series of temporal filters intended to smooth the trajectories in our time series by looking for short-term land-cover changes that are temporally unlikely.
We applied five-year and eight-year moving window filters designed to search for short periods of land cover classifications that do not match those immediately before and after, and subsequently update them to match the surrounding classifications.
Specifically, the five-year filter searched for one year periods that did not match the two years immediately before and after (i.e.~patterns of 11011, where 1 represents non-cropland and 0 represents cropland), and our eight-year filter searched for two year periods that did not match the three years before and after (i.e.~11100111).
The central classifications were then updated to match the classes on either end.
Together with our five year abandonment threshold, these temporal filters have the ultimate effect of minimizing the effect of very short-term misclassifications that would otherwise look like recultivation.\footnote{Note to David and Volker: the results presented here (from run date: \_2021\_03\_13), use only these regular temporal filters. We also considered applying modified versions of these moving window filters to the start of the time series, but decided against it, for the sake of consistency and parsimony.}

\hypertarget{calculating-abandonment-length}{%
\subsection{Calculating abandonment length}\label{calculating-abandonment-length}}

We calculate abandonment length as the time (in years) between the year a pixel first transitions from cropland to non-cropland vegetation and the year it transitions back and is classified as recultivated.
Given our abandonment threshold, the minimum abandonment length is five years.
Because a given pixel may go through multiple distinct periods of abandonment (being abandoned and recultivated multiple times) throughout the time series, we calculate the mean abandonment length in two ways: 1) incorporating all distinct periods of abandonment, and 2) considering only the longest period of abandonment for each pixel (``maximum abandonment length'') (Figure \ref{fig:mean-abn-length}).
We calculated area using the \texttt{raster::area()} function.

\hypertarget{tracking-abandonment-trajectories}{%
\subsubsection{Tracking abandonment trajectories}\label{tracking-abandonment-trajectories}}

In order to track the persistence of individual \ldots{}

This will involve the code in \texttt{cc\_summarize\_abn\_dts()} and \texttt{cc\_calc\_persistence()}.
Write a simple sentence describing what we did here.

\hypertarget{modeling-abandonment-decay}{%
\subsection{Modeling abandonment decay}\label{modeling-abandonment-decay}}

\hypertarget{motivation}{%
\subsubsection{Motivation}\label{motivation}}

\begin{quote}
How does the average length of time abandoned differ from the decay rate, and what is the importance of looking at decay rates?
\end{quote}

The mean length of time abandoned (in years) is based on all periods of abandonment in the time series, because a given pixel can experience multiple periods of abandonment throughout the time series.
This value tells us about the general persistence of abandoned land at a given site, over the course of the full time series.
However, this value is limited by when the majority of the abandonment took place at a site, because we have now way of knowing how long abandonment that takes place towards the end of the time series will last for.
Will it last for two additional years beyond the end of the time series, or 20?

As a result, the mean abandonment length does not tell us how long to expect a piece of land to remain abandoned, and does not tell us about how abandonment length varies through time.
(Note: assessing temporal changes in mean length abandoned would involve looking at the mean length of time abandoned at each year of Figure \ref{fig:abn-panel-s}a.)
Restricting the mean length to only those pixels that are both abandoned and recultivated within the time series would exclude a large portion of our data, and may underestimate the eventual lengths of many periods of abandonment that begin close to the end of our time series.

To address this challenge, we look at groups of pixels that are abandoned in a given year (which we call ``cohorts'' of abandonment) and track their trajectories through time.
For each cohort of abandoned land, we track recultivation, or ``abandonment decay,'' by assessing the proportion of each cohort that remains abandoned through time, relative to when that land was first abandoned (the cohort year).
Decay rates provide information about how long it takes for land to be recultivated, complementing the mean abandonment length and providing a more nuanced story about how long to expect abandonment to last.

For example, a site may have a relatively short mean length of abandonment (e.g.~Shaanxi/Shanxi Province, with a mean abandonment length of 13 years; see Figure \ref{fig:site-locations}), but also have a gradual decay rate, indicating that land should stay abandoned for a relatively longer amount of time.
This may result from more abandonment occurring towards the end of the time series; this land simply does not have as long to age and shows up as younger in our data, regardless of how long it may last.

Looking at abandonment decay rates for each cohort individually allows us to produce a decay rate for each site in general in a way that accounts for when during the time series a piece of land was abandoned (i.e.~giving us a sense of how long to expect a given piece of land to remain abandoned, even into the future).

Importantly, this approach also allows us to look at changes in persistence over time.
We are able to see if the rate at which abandonment land decays (i.e.~is recultivated) gets faster, stays the same, or slows down over time.

Translating decay rates into an intuitive metric is difficult. Many studies of exponential decay reference the ``half-life,'' or the time it takes for a quantity to reduce by half. Because we model abandonment decay flexibly, so that the rate at which a given cohort of land is recultivated can speed up or slow down through time, this concept of half-life does not work in our context. Rather, we measure the length of time it takes for a given cohort to decay to a specific proportion: the time it takes for a cohort to fall to a proportion of 50\%.

\hypertarget{decay-model-specification}{%
\subsubsection{Decay model specification}\label{decay-model-specification}}

To investigate the decay of abandoned land, we ran linear models using the \texttt{stats::lm()} function in R's core statistics package, predicting the proportion of the initial amount of abandoned land remaining abandoned as a function of time since initial abandonment.
We tested a range of simple model specifications, including linear and log transformations of both \emph{proportion} and \emph{time}.
Due to a linear relationship between model residuals and time when including only one term for \emph{time}, also tested models containing multiple \emph{time} predictor terms, including log and linear terms.
Importantly, we also include \emph{cohort} fixed effects, allowing for the estimation of unique coefficients for each cohort of abandoned land.

We compared the quality of each model using Akaike Information Criterion (AIC) values and chose the highest performing model specification, which predicted \emph{proportion} with one log-transformed and one linear term of \emph{time} since abandonment (see Figure \ref{fig:AIC} and Equation \eqref{eq:mod-spec}).

Model assumptions were tested through visual inspection of fitted values vs.~residuals plots and qq plots.
Full details are contained in Section \ref{mod-AIC-diag}.

\hypertarget{data-availability-statement}{%
\subsection{Data Availability Statement}\label{data-availability-statement}}

Code to replicate these analyses is available on GitHub at \url{https://github.com/chriscra/abandonment_trajectories}, and a static archive can be found at \href{https://doi.org/}{zenodo}.

\hypertarget{acknowledgements}{%
\section{Acknowledgements}\label{acknowledgements}}

Our analyses were supported by Princeton University's Research Computing cluster resources. (\emph{Insert text here from PRC website})

\newpage

\hypertarget{references}{%
\section{References}\label{references}}

\hypertarget{refs}{}
\begin{CSLReferences}{1}{0}
\leavevmode\hypertarget{ref-Aide2019}{}%
Aide, T. Mitchell, H. Ricardo Grau, Jordan Graesser, Maria Jose Andrade-Nuñez, Ezequiel Aráoz, Ana P. Barros, Marconi Campos-Cerqueira, et al. 2019. {``{Woody vegetation dynamics in the tropical and subtropical Andes from 2001 to 2014: Satellite image interpretation and expert validation}.''} \emph{Global Change Biology} 25 (6): 2112--26. \url{https://doi.org/10.1111/gcb.14618}.

\leavevmode\hypertarget{ref-Alcantara2013}{}%
Alcantara, Camilo, Tobias Kuemmerle, Matthias Baumann, Eugenia V. Bragina, Patrick Griffiths, Patrick Hostert, Jan Knorn, et al. 2013. {``{Mapping the extent of abandoned farmland in Central and Eastern Europe using MODIS time series satellite data}.''} \emph{Environmental Research Letters} 8 (3). \url{https://doi.org/10.1088/1748-9326/8/3/035035}.

\leavevmode\hypertarget{ref-R-rmarkdown}{}%
Allaire, J J, Yihui Xie, Jonathan McPherson, Javier Luraschi, Kevin Ushey, Aron Atkins, Hadley Wickham, Joe Cheng, Winston Chang, and Richard Iannone. 2020. \emph{{rmarkdown: Dynamic Documents for R}}. \url{https://cran.r-project.org/package=rmarkdown}.

\leavevmode\hypertarget{ref-Campbell2008}{}%
Campbell, J. Elliott, David B. Lobell, Robert C. Genova, and Christopher B. Field. 2008. {``{The global potential of bioenergy on abandoned agriculture lands}.''} \emph{Environmental Science and Technology} 42 (15): 5791--94. \url{https://doi.org/10.1021/es800052w}.

\leavevmode\hypertarget{ref-Chazdon2020}{}%
Chazdon, Robin L., David Lindenmayer, Manuel R. Guariguata, Renato Crouzeilles, José María Rey Benayas, and Elena Lazos Chavero. 2020. {``{Fostering natural forest regeneration on former agricultural land through economic and policy interventions}.''} \emph{Environmental Research Letters} 15 (4): 043002. \url{https://doi.org/10.1088/1748-9326/ab79e6}.

\leavevmode\hypertarget{ref-Crouzeilles2020}{}%
Crouzeilles, Renato, Hawthorne L. Beyer, Lara M. Monteiro, Rafael Feltran-Barbieri, Ana C. M. Pessôa, Felipe S. M. Barros, David B. Lindenmayer, et al. 2020. {``{Achieving cost‐effective landscape‐scale forest restoration through targeted natural regeneration}.''} \emph{Conservation Letters}, no. February (February): 1--9. \url{https://doi.org/10.1111/conl.12709}.

\leavevmode\hypertarget{ref-R-data.table}{}%
Dowle, Matt, and Arun Srinivasan. 2021. \emph{Data.table: Extension of `Data.frame`}.

\leavevmode\hypertarget{ref-Estel2015}{}%
Estel, Stephan, Tobias Kuemmerle, Camilo Alcántara, Christian Levers, Alexander Prishchepov, and Patrick Hostert. 2015. {``{Mapping farmland abandonment and recultivation across Europe using MODIS NDVI time series}.''} \emph{Remote Sensing of Environment} 163: 312--25. \url{https://doi.org/10.1016/j.rse.2015.03.028}.

\leavevmode\hypertarget{ref-R-raster}{}%
Hijmans, Robert J. 2020. \emph{Raster: Geographic Data Analysis and Modeling}. \url{https://rspatial.org/raster}.

\leavevmode\hypertarget{ref-Isbell2019}{}%
Isbell, Forest, David Tilman, Peter B Reich, and Adam Thomas Clark. 2019. {``{Deficits of biodiversity and productivity linger a century after agricultural abandonment}.''} \emph{Nature Ecology {\&} Evolution} 3 (11): 1533--38. \url{https://doi.org/10.1038/s41559-019-1012-1}.

\leavevmode\hypertarget{ref-Lark2015}{}%
Lark, Tyler J, J Meghan Salmon, and Holly K Gibbs. 2015. {``{Cropland expansion outpaces agricultural and biofuel policies in the United States}.''} \emph{Environmental Research Letters} 10 (4): 044003. \url{https://doi.org/10.1088/1748-9326/10/4/044003}.

\leavevmode\hypertarget{ref-Li2018}{}%
Li, Shengfa, Xiubin Li, Laixiang Sun, Guiying Cao, Guenther Fischer, and Sylvia Tramberend. 2018. {``{An estimation of the extent of cropland abandonment in mountainous regions of China}.''} \emph{Land Degradation and Development} 29 (5): 1327--42. \url{https://doi.org/10.1002/ldr.2924}.

\leavevmode\hypertarget{ref-Munroe2013}{}%
Munroe, Darla K., Derek B. van Berkel, Peter H. Verburg, and Jeffrey L. Olson. 2013. {``{Alternative trajectories of land abandonment: Causes, consequences and research challenges}.''} \emph{Current Opinion in Environmental Sustainability} 5 (5): 471--76. \url{https://doi.org/10.1016/j.cosust.2013.06.010}.

\leavevmode\hypertarget{ref-Nunes2020}{}%
Nunes, Sâmia, Luis Oliveira, João Siqueira, Douglas C Morton, and Carlos M Souza. 2020. {``{Unmasking secondary vegetation dynamics in the Brazilian Amazon}.''} \emph{Environmental Research Letters} 15 (3): 034057. \url{https://doi.org/10.1088/1748-9326/ab76db}.

\leavevmode\hypertarget{ref-Naess2021}{}%
Næss, Jan Sandstad, Otavio Cavalett, and Francesco Cherubini. 2021. {``{The land--energy--water nexus of global bioenergy potentials from abandoned cropland}.''} \emph{Nature Sustainability}, January. \url{https://doi.org/10.1038/s41893-020-00680-5}.

\leavevmode\hypertarget{ref-Ramankutty1999}{}%
Ramankutty, Navin, and Jonathan A. Foley. 1999. {``{Estimating Historical Changes in Land Cover: North American Croplands from 1850-1992}.''} \emph{Global Ecology and Biogeography} 8 (5): 381--96.

\leavevmode\hypertarget{ref-Reid2018}{}%
Reid, J. Leighton, Matthew E. Fagan, James Lucas, Joshua Slaughter, and Rakan A. Zahawi. 2018. {``{The ephemerality of secondary forests in southern Costa Rica}.''} \emph{Conservation Letters}, no. September: e12607. \url{https://doi.org/10.1111/conl.12607}.

\leavevmode\hypertarget{ref-RStudioTeam2020}{}%
RStudio Team. 2020. {``{RStudio: Integrated Development Environment for R}.''} Boston, MA: RStudio, Inc. \url{http://www.rstudio.com/}.

\leavevmode\hypertarget{ref-Smith2020}{}%
Smith, Charlotte C, Fernando Del Bon Espírito‐Santo, John R Healey, Paul J. Young, Gareth D. Lennox, Joice Ferreira, and Jos Barlow. 2020. {``{Secondary forests offset less than 10{\%} of deforestation‐mediated carbon emissions in the Brazilian Amazon}.''} \emph{Global Change Biology} 6 (September): gcb.15352. \url{https://doi.org/10.1111/gcb.15352}.

\leavevmode\hypertarget{ref-R-bookdown}{}%
Xie, Yihui. 2020. \emph{{bookdown: Authoring Books and Technical Documents with R Markdown}}. \url{https://cran.r-project.org/package=bookdown}.

\leavevmode\hypertarget{ref-R-knitr}{}%
---------. 2021. \emph{Knitr: A General-Purpose Package for Dynamic Report Generation in r}. \url{https://yihui.org/knitr/}.

\leavevmode\hypertarget{ref-Yin2020}{}%
Yin, He, Amintas Brandão, Johanna Buchner, David Helmers, Benjamin G Iuliano, Niwaeli E Kimambo, Katarzyna E. Lewińska, et al. 2020. {``{Monitoring cropland abandonment with Landsat time series}.''} \emph{Remote Sensing of Environment} 246 (September): 111873. \url{https://doi.org/10.1016/j.rse.2020.111873}.

\leavevmode\hypertarget{ref-Yu2018}{}%
Yu, Zhen, and Chaoqun Lu. 2018. {``{Historical cropland expansion and abandonment in the continental U.S. during 1850 to 2016}.''} \emph{Global Ecology and Biogeography} 27 (3): 322--33. \url{https://doi.org/10.1111/geb.12697}.

\leavevmode\hypertarget{ref-Yu2019}{}%
Yu, Zhen, Chaoqun Lu, Hanqin Tian, and Josep G. Canadell. 2019. {``{Largely underestimated carbon emission from land use and land cover change in the conterminous United States}.''} \emph{Global Change Biology} 25 (11): 3741--52. \url{https://doi.org/10.1111/gcb.14768}.

\end{CSLReferences}

\newpage

\hypertarget{supporting-information}{%
\section{Supporting Information}\label{supporting-information}}

\hypertarget{extended-methods}{%
\subsection{Extended Methods}\label{extended-methods}}

\hypertarget{extended-basic-results}{%
\subsection{Extended basic results}\label{extended-basic-results}}

\hypertarget{site-locations}{%
\subsubsection{Site locations}\label{site-locations}}

Alternative location of the site map here.

\hypertarget{area-of-each-site}{%
\subsubsection{Area of each site}\label{area-of-each-site}}

The area abandoned at each site is shown in Figure \ref{fig:area-abn-panel}.
Area by land cover class is shown for each site in Figure \ref{fig:area-by-lc}, and the area abandoned is shared by age class in Figure \ref{fig:area-abn-by-age-class}.



\begin{figure}
\includegraphics[width=1\textwidth]{/Users/christophercrawford/Google Drive/_Projects/abandonment_trajectories/output/plots/_2021_03_13/area_abn_panel_2021_03_13} \caption{Area abandoned a) as of 2017, b) at any point between 1987-2017, c) as a proportion of the total site area, and d) as a proportion of the total cropland area in 1987. Note that sites are shown in ascending order of area as a proportion of total site area.}\label{fig:area-abn-panel}
\end{figure}



\begin{figure}
\includegraphics[width=1\textwidth]{/Users/christophercrawford/Google Drive/_Projects/abandonment_trajectories/output/plots/_2021_03_13/area_by_lc_grid_2021_03_13} \caption{Area in each land cover class at each site through time. Land cover classes are cropland, grassland, woody vegetation, non-vegetation, and abandoned (for at least 5 years).}\label{fig:area-by-lc}
\end{figure}



\begin{figure}
\includegraphics[width=1\textwidth]{/Users/christophercrawford/Google Drive/_Projects/abandonment_trajectories/output/plots/_2021_03_13/abn_area_by_age_grid_2021_03_13} \caption{Area abandoned at each site, shown according to the age of each abandoned field (in years).}\label{fig:area-abn-by-age-class}
\end{figure}

\hypertarget{abn-thresholds}{%
\subsubsection{The effect of varying abandonment definitions}\label{abn-thresholds}}

\begin{itemize}
\tightlist
\item
  Abandonment threshold, including a higher threshold (something like 6 or 10 years), along with how thresholds change total pixels/area.
\end{itemize}



\begin{figure}
\includegraphics[width=1\textwidth]{/Users/christophercrawford/Google Drive/_Projects/abandonment_trajectories/output/plots/_2021_03_13/mean_lengths_by_threshold_2021_03_13} \caption{Mean abandonment lengths shown for various abandonment thresholds.}\label{fig:abn-thresholds-mean-duration}
\end{figure}

Like the mean duration of abandonment, the amount of recultivation depends on our definition of abandonment (Figure \ref{fig:recult-by-threshold}). We find that the mean area of abandoned croplands that get recultivated declines from 37.77\% with a 5 year threshold to 30.93\% with a 7 year threshold, and 22.83\% with a 10 year threshold.

However, we also find that a significant area of these abandoned croplands were recultivated by the end of our time series (on average of abandoned area across sites).



\begin{figure}
\includegraphics[width=1\textwidth]{/Users/christophercrawford/Google Drive/_Projects/abandonment_trajectories/output/plots/_2021_03_13/recultivation_rates_by_threshold_2021_03_13} \caption{Recultivation rates shown for various abandonment thresholds.}\label{fig:recult-by-threshold}
\end{figure}



\begin{figure}
\includegraphics[width=1\textwidth]{/Users/christophercrawford/Google Drive/_Projects/abandonment_trajectories/output/plots/_2021_03_13/abn_prop_lc_2017_2021_03_13} \caption{Proportion of abandoned cropland in each land cover class as of 2017.}\label{fig:abn-prop-lc}
\end{figure}

\hypertarget{summary-composite-figures}{%
\subsubsection{Summary composite figures}\label{summary-composite-figures}}













\begin{figure}
\includegraphics[width=1\textwidth]{/Users/christophercrawford/Google Drive/_Projects/abandonment_trajectories/output/plots/_2021_03_13/si_panel_2021_03_13_b} \caption{Abandonment patterns for Eastern Belarus/Smolensk, Russia.}\label{fig:panel-b}
\end{figure}

\begin{figure}
\includegraphics[width=1\textwidth]{/Users/christophercrawford/Google Drive/_Projects/abandonment_trajectories/output/plots/_2021_03_13/si_panel_2021_03_13_bh} \caption{Abandonment patterns for Bosnia \& Herzegovina.}\label{fig:panel-bh}
\end{figure}

\begin{figure}
\includegraphics[width=1\textwidth]{/Users/christophercrawford/Google Drive/_Projects/abandonment_trajectories/output/plots/_2021_03_13/si_panel_2021_03_13_c} \caption{Abandonment patterns for Chongqing Province, China.}\label{fig:panel-c}
\end{figure}

\begin{figure}
\includegraphics[width=1\textwidth]{/Users/christophercrawford/Google Drive/_Projects/abandonment_trajectories/output/plots/_2021_03_13/si_panel_2021_03_13_g} \caption{Abandonment patterns for Goiás, Brazil.}\label{fig:panel-g}
\end{figure}

\begin{figure}
\includegraphics[width=1\textwidth]{/Users/christophercrawford/Google Drive/_Projects/abandonment_trajectories/output/plots/_2021_03_13/si_panel_2021_03_13_i} \caption{Abandonment patterns for Iraq.}\label{fig:panel-i}
\end{figure}

\begin{figure}
\includegraphics[width=1\textwidth]{/Users/christophercrawford/Google Drive/_Projects/abandonment_trajectories/output/plots/_2021_03_13/si_panel_2021_03_13_mg} \caption{Abandonment patterns for Mato Grosso, Brazil.}\label{fig:panel-mg}
\end{figure}

\begin{figure}
\includegraphics[width=1\textwidth]{/Users/christophercrawford/Google Drive/_Projects/abandonment_trajectories/output/plots/_2021_03_13/si_panel_2021_03_13_n} \caption{Abandonment patterns for Nebraska, USA.}\label{fig:panel-n}
\end{figure}

\begin{figure}
\includegraphics[width=1\textwidth]{/Users/christophercrawford/Google Drive/_Projects/abandonment_trajectories/output/plots/_2021_03_13/si_panel_2021_03_13_o} \caption{Abandonment patterns for Orenburg, Russia.}\label{fig:panel-o}
\end{figure}

\begin{figure}
\includegraphics[width=1\textwidth]{/Users/christophercrawford/Google Drive/_Projects/abandonment_trajectories/output/plots/_2021_03_13/si_panel_2021_03_13_s} \caption{Abandonment patterns for Shaanxi/Shanxi Provinces, China.}\label{fig:panel-s}
\end{figure}

\begin{figure}
\includegraphics[width=1\textwidth]{/Users/christophercrawford/Google Drive/_Projects/abandonment_trajectories/output/plots/_2021_03_13/si_panel_2021_03_13_v} \caption{Abandonment patterns for Volgograd, Russia.}\label{fig:panel-v}
\end{figure}

\begin{figure}
\includegraphics[width=1\textwidth]{/Users/christophercrawford/Google Drive/_Projects/abandonment_trajectories/output/plots/_2021_03_13/si_panel_2021_03_13_w} \caption{Abandonment patterns for Wisconsin, USA.}\label{fig:panel-w}
\end{figure}

\hypertarget{maps}{%
\subsubsection{Maps}\label{maps}}

May not be necessary, since they will be the focus of Chapter 4.

Maps of a) abandonment \emph{age in 2017} (Figure \ref{fig:plot-age-in-2017}) and b) \emph{max age} for all sites (Figure \ref{fig:plot-max-age}).

See function \texttt{cc\_age\_levelplot\_hist()} to produce these side by side, with site name in the title



\begin{figure}
\includegraphics[width=1\textwidth]{/Users/christophercrawford/Google Drive/_Projects/abandonment_trajectories/output/plots/_2021_03_13/spatial_reg/age_duration_2017_w_dist_2021_03_13_s} \caption{Duration of abandonment (in years) as of 2017, Shaanxi/Shanxi Provinces, China.}\label{fig:plot-age-in-2017}
\end{figure}



\begin{figure}
\includegraphics[width=1\textwidth]{/Users/christophercrawford/Google Drive/_Projects/abandonment_trajectories/output/plots/_2021_03_13/spatial_reg/age_duration_max_w_dist_2021_03_13_s} \caption{Maximum duration of abandonment (in years), Shaanxi/Shanxi Provinces, China.}\label{fig:plot-max-age}
\end{figure}

\hypertarget{decay-models}{%
\subsection{Decay models}\label{decay-models}}



\begin{figure}
\includegraphics[width=1\textwidth]{/Users/christophercrawford/Google Drive/_Projects/abandonment_trajectories/output/plots/_2021_03_13/decay/decay_mod_grid_2021_03_13} \caption{Decay model results for all sites, showing raw data (green dots) and modeled abandonment trajectories for individual years (green lines). The mean decay trajectory for the site is calculated as the mean of the coefficients for each cohort (solid blue line). Larger model results are shown in Figures \ref{fig:decay-model-indiv-site-b}-\ref{fig:decay-model-indiv-site-w}.}\label{fig:decay-model-grid}
\end{figure}

\hypertarget{mod-AIC-diag}{%
\subsubsection{Model comparisons and diagnostics}\label{mod-AIC-diag}}

Model diagnostics:

\begin{itemize}
\tightlist
\item
  residuals vs.~fitted values (for each site)
\item
  qqplot (for each site)
\item
  residuals vs.~time (all sites)
\item
  AIC
\end{itemize}

We chose a model with the following specifications shown in Equation \eqref{eq:mod-spec}.
For cohorts of abandonment initially abandoned in years \(y = 1988 ... 2013\), we estimate the proportion \(p\) of each cohort \(y\) remaining abandoned as a function of time \(t\) (i.e.~based on the number of years after initial abandonment).

\begin{equation}
p_{y} = 1 + \beta_{1,y} log(t + 1) + \beta_{2,y} t \label{eq:mod-spec}
\end{equation}

Where \(\beta_{1,y}\) represents the regression coefficient on the log term of time \(t\) for cohort \(y\), and \(\beta_{2,y}\) represents the regression coefficient on the linear term of time \(t\) for cohort \(y\).

This corresponds to a \texttt{lm()} call of \texttt{lm(formula\ =\ I(proportion\ -\ 1)\ \ \textasciitilde{}\ 0\ +\ log(time\ +\ 1):cohort\ +\ I(time):cohort)}



\begin{figure}
\includegraphics[width=1\textwidth]{/Users/christophercrawford/Google Drive/_Projects/abandonment_trajectories/output/plots/_2021_03_13/decay/mod_AIC_mega_plot_2021_03_13} \caption{Absolute value of AIC values for various model specifications tested. Greater absolute values indicate a better model fit.}\label{fig:AIC}
\end{figure}



Model results:

\begin{itemize}
\tightlist
\item
  Decay rate rate of change for each site, in four quadrants
\item
  Single site: decay rates through time, with linear model trend shown.
\item
  Projected proportions remaining after set periods of time (10 years, 20 years, 30 years, etc.)
\item
  Decay curves with model fitted values for each site.
\item
  an alternative representation of the mean decay rate (Figure \ref{fig:decay-time-to-range})
\end{itemize}



\begin{figure}
\includegraphics[width=1\textwidth]{/Users/christophercrawford/Google Drive/_Projects/abandonment_trajectories/output/plots/_2021_03_13/decay/time_to_range_all_l3_2021_03_13} \caption{An alternative representation of the mean decay rate for each site (also shown in Figure \ref{fig:decay-curves-by-site}). The color of each dot corresponds to the proportion remaining after a given amount of time.}\label{fig:decay-time-to-range}
\end{figure}

\hypertarget{individual-decay-model-results}{%
\subsubsection{Individual Decay Model Results}\label{individual-decay-model-results}}























\begin{figure}
\includegraphics[width=1\textwidth]{/Users/christophercrawford/Google Drive/_Projects/abandonment_trajectories/output/plots/_2021_03_13/decay/l3_cohort_plus_2021_03_13_b} \caption{Decay model results for Eastern Belarus and Smolensk, Russia, showing raw data (green dots), modeled abandonment trajectories for individual years (green lines), and the mean decay trajectory for the site as a whole (blue lines). The mean decay trajectory is calculated in two ways: 1) from the mean of the coefficients for each cohort (solid blue line), and 2) as a single linear regression without cohort-level fixed effects (F.E.; dashed blue line).}\label{fig:decay-model-indiv-site-b}
\end{figure}

\begin{figure}
\includegraphics[width=1\textwidth]{/Users/christophercrawford/Google Drive/_Projects/abandonment_trajectories/output/plots/_2021_03_13/decay/l3_cohort_plus_2021_03_13_bh} \caption{Decay model results for Bosnia \& Herzegovina, showing raw data (green dots), modeled abandonment trajectories for individual years (green lines), and the mean decay trajectory for the site as a whole (blue lines). The mean decay trajectory is calculated in two ways: 1) from the mean of the coefficients for each cohort (solid blue line), and 2) as a single linear regression without cohort-level fixed effects (F.E.; dashed blue line).}\label{fig:decay-model-indiv-site-bh}
\end{figure}

\begin{figure}
\includegraphics[width=1\textwidth]{/Users/christophercrawford/Google Drive/_Projects/abandonment_trajectories/output/plots/_2021_03_13/decay/l3_cohort_plus_2021_03_13_c} \caption{Decay model results for Chongqing Province, China, showing raw data (green dots), modeled abandonment trajectories for individual years (green lines), and the mean decay trajectory for the site as a whole (blue lines). The mean decay trajectory is calculated in two ways: 1) from the mean of the coefficients for each cohort (solid blue line), and 2) as a single linear regression without cohort-level fixed effects (F.E.; dashed blue line).}\label{fig:decay-model-indiv-site-c}
\end{figure}

\begin{figure}
\includegraphics[width=1\textwidth]{/Users/christophercrawford/Google Drive/_Projects/abandonment_trajectories/output/plots/_2021_03_13/decay/l3_cohort_plus_2021_03_13_g} \caption{Decay model results for Goiás, Brazil, showing raw data (green dots), modeled abandonment trajectories for individual years (green lines), and the mean decay trajectory for the site as a whole (blue lines). The mean decay trajectory is calculated in two ways: 1) from the mean of the coefficients for each cohort (solid blue line), and 2) as a single linear regression without cohort-level fixed effects (F.E.; dashed blue line).}\label{fig:decay-model-indiv-site-g}
\end{figure}

\begin{figure}
\includegraphics[width=1\textwidth]{/Users/christophercrawford/Google Drive/_Projects/abandonment_trajectories/output/plots/_2021_03_13/decay/l3_cohort_plus_2021_03_13_i} \caption{Decay model results for Iraq, showing raw data (green dots), modeled abandonment trajectories for individual years (green lines), and the mean decay trajectory for the site as a whole (blue lines). The mean decay trajectory is calculated in two ways: 1) from the mean of the coefficients for each cohort (solid blue line), and 2) as a single linear regression without cohort-level fixed effects (F.E.; dashed blue line).}\label{fig:decay-model-indiv-site-i}
\end{figure}

\begin{figure}
\includegraphics[width=1\textwidth]{/Users/christophercrawford/Google Drive/_Projects/abandonment_trajectories/output/plots/_2021_03_13/decay/l3_cohort_plus_2021_03_13_mg} \caption{Decay model results for Mato Grosso, Brazil, showing raw data (green dots), modeled abandonment trajectories for individual years (green lines), and the mean decay trajectory for the site as a whole (blue lines). The mean decay trajectory is calculated in two ways: 1) from the mean of the coefficients for each cohort (solid blue line), and 2) as a single linear regression without cohort-level fixed effects (F.E.; dashed blue line).}\label{fig:decay-model-indiv-site-mg}
\end{figure}

\begin{figure}
\includegraphics[width=1\textwidth]{/Users/christophercrawford/Google Drive/_Projects/abandonment_trajectories/output/plots/_2021_03_13/decay/l3_cohort_plus_2021_03_13_n} \caption{Decay model results for Nebraska, USA, showing raw data (green dots), modeled abandonment trajectories for individual years (green lines), and the mean decay trajectory for the site as a whole (blue lines). The mean decay trajectory is calculated in two ways: 1) from the mean of the coefficients for each cohort (solid blue line), and 2) as a single linear regression without cohort-level fixed effects (F.E.; dashed blue line).}\label{fig:decay-model-indiv-site-n}
\end{figure}

\begin{figure}
\includegraphics[width=1\textwidth]{/Users/christophercrawford/Google Drive/_Projects/abandonment_trajectories/output/plots/_2021_03_13/decay/l3_cohort_plus_2021_03_13_o} \caption{Decay model results for Orenburg, Russia, showing raw data (green dots), modeled abandonment trajectories for individual years (green lines), and the mean decay trajectory for the site as a whole (blue lines). The mean decay trajectory is calculated in two ways: 1) from the mean of the coefficients for each cohort (solid blue line), and 2) as a single linear regression without cohort-level fixed effects (F.E.; dashed blue line).}\label{fig:decay-model-indiv-site-o}
\end{figure}

\begin{figure}
\includegraphics[width=1\textwidth]{/Users/christophercrawford/Google Drive/_Projects/abandonment_trajectories/output/plots/_2021_03_13/decay/l3_cohort_plus_2021_03_13_s} \caption{Decay model results for Shaanxi/Shanxi Provinces, China, showing raw data (green dots), modeled abandonment trajectories for individual years (green lines), and the mean decay trajectory for the site as a whole (blue lines). The mean decay trajectory is calculated in two ways: 1) from the mean of the coefficients for each cohort (solid blue line), and 2) as a single linear regression without cohort-level fixed effects (F.E.; dashed blue line).}\label{fig:decay-model-indiv-site-s}
\end{figure}

\begin{figure}
\includegraphics[width=1\textwidth]{/Users/christophercrawford/Google Drive/_Projects/abandonment_trajectories/output/plots/_2021_03_13/decay/l3_cohort_plus_2021_03_13_v} \caption{Decay model results for Volgograd, Russia, showing raw data (green dots), modeled abandonment trajectories for individual years (green lines), and the mean decay trajectory for the site as a whole (blue lines). The mean decay trajectory is calculated in two ways: 1) from the mean of the coefficients for each cohort (solid blue line), and 2) as a single linear regression without cohort-level fixed effects (F.E.; dashed blue line).}\label{fig:decay-model-indiv-site-v}
\end{figure}

\begin{figure}
\includegraphics[width=1\textwidth]{/Users/christophercrawford/Google Drive/_Projects/abandonment_trajectories/output/plots/_2021_03_13/decay/l3_cohort_plus_2021_03_13_w} \caption{Decay model results for Wisconsin, USA, showing raw data (green dots), modeled abandonment trajectories for individual years (green lines), and the mean decay trajectory for the site as a whole (blue lines). The mean decay trajectory is calculated in two ways: 1) from the mean of the coefficients for each cohort (solid blue line), and 2) as a single linear regression without cohort-level fixed effects (F.E.; dashed blue line).}\label{fig:decay-model-indiv-site-w}
\end{figure}

\hypertarget{section-extrapolation-si}{%
\subsection{Extrapolating abandonment duration}\label{section-extrapolation-si}}

\hypertarget{alternative-assumptions-about-annual-additional-abandonment}{%
\subsubsection{Alternative assumptions about annual additional abandonment}\label{alternative-assumptions-about-annual-additional-abandonment}}

However, these assumptions may not be very realistic (actually, they almost certainly are not realistic).
In particular, it is unlikely that a constant amount of land will be newly abandoned each year indefinitely.
I can think of three alternative assumptions we could make:

\begin{enumerate}
\def\labelenumi{\arabic{enumi}.}
\tightlist
\item
  We could assume that the amount of abandoned land each year would be more likely to gradually decline, so that less and less land is abandoned each year.
\item
  We could also set a limit to how much total area could be abandoned, perhaps based on a percent of cropland in a site, for example.
\item
  Or, finally, we could base this on the trend in abandoned land each year (running a regression on the gain columns in part d of the main results panel figures).
\end{enumerate}

\emph{To do: run this extrapolation for the three different sets of assumptions, and assess the results.}

However, I did it this simple way first, just to illuminate what these trends would mean if they were taken to the extreme.
And, the extrapolation shows some pretty striking results: many of the sites do not see any land last longer than 20 years.
The average age of abandonment rarely rises above 20 years, which really isn't that long in terms of carbon storage or biodiversity recovery.

This is not the case in the time series data we have, mostly because at many of the sites, the newer cohorts of land decay more quickly than the older ones.
Perhaps this is a sign that the least valuable land is abandoned earliest, and therefore is most durable.
The decay rates for cohorts towards the end of the time series are rather fast at many sites (these cohorts also have the fewest observations), which makes the site average fast as well.
This is one reason why the mean age is so low in many places.

\hypertarget{alternative-mean-decay-rates}{%
\subsubsection{Alternative mean decay rates}\label{alternative-mean-decay-rates}}

Currently, to get the mean decay rate for each site, I simply average each coefficient (for the log and the linear terms) across all cohorts, and then use these two mean values as the coefficients for the mean site trend (the solid blue lines in Figure \ref{fig:decay-model-grid}).
I think this makes sense, because it treats each cohort of abandonment equally (rather than giving more weight to cohorts with more observations, i.e.~older cohorts).
Having cohort fixed effects allows for a unique decay rate for each year of abandonment, which makes the models fit very well.
But, the fits for the recent cohorts with few observations are quite straight.

Alternatively, however, I can also just run a single regression without the cohort fixed effects, which produces a single estimate for the log and linear coefficients, weighting all points equally (therefore giving more weight to older cohorts with more points; the dashed blue lines in Figure \ref{fig:decay-model-grid}).

I can then conduct the extrapolation based on the decay rates derived from that single linear regression for each site (in other words, across all cohorts of abandonment).
These extrapolation results are shown below in Figures \ref{fig:extrapolation-combo-no-fe} and \ref{fig:extrapolation-area-by-age-no-fe}.

However, as you can see, some of the sites have a decay curve that clearly doesn't make sense when extrapolated out (eventually curving upwards over time).
As a result, I think we should stick with our original method of calculating the mean trend at each site (averaging across the coefficients for each cohort), but I just wanted to call it to your attention.



\begin{figure}
\includegraphics[width=1\textwidth]{/Users/christophercrawford/Google Drive/_Projects/abandonment_trajectories/output/plots/_2021_03_13/decay/extrapolation_combo_no_fe_tall_2021_03_13} \caption{The results of our simple extrapolation, but assuming a decay trend for each site calculated without cohort fixed effects (a), resulting in b) the mean age of abandonment, and c) the total area abandoned into the future. Colors corresponding to each site are consistent across the three panels.}\label{fig:extrapolation-combo-no-fe}
\end{figure}







\begin{figure}
\includegraphics[width=1\textwidth]{/Users/christophercrawford/Google Drive/_Projects/abandonment_trajectories/output/plots/_2021_03_13/decay/extrapolate_age_by_age_bin_no_fe_2021_03_13} \caption{Extrapolating area by age bin, assuming site decay rates derived from linear models that do not use cohort fixed effects (treating each observation equally).}\label{fig:extrapolation-area-by-age-no-fe}
\end{figure}

\end{document}
