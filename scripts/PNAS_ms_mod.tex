\documentclass[9pt,twocolumn,twoside,]{pnas-new}

%% Some pieces required from the pandoc template
\providecommand{\tightlist}{%
  \setlength{\itemsep}{0pt}\setlength{\parskip}{0pt}}

% Use the lineno option to display guide line numbers if required.
% Note that the use of elements such as single-column equations
% may affect the guide line number alignment.


\usepackage[T1]{fontenc}
\usepackage[utf8]{inputenc}

% Pandoc citation processing
\newlength{\csllabelwidth}
\setlength{\csllabelwidth}{3em}
\newlength{\cslhangindent}
\setlength{\cslhangindent}{1.5em}
% for Pandoc 2.8 to 2.10.1
\newenvironment{cslreferences}%
  {}%
  {\par}
% For Pandoc 2.11+
\newenvironment{CSLReferences}[2] % #1 hanging-ident, #2 entry spacing
 {% don't indent paragraphs
  \setlength{\parindent}{0pt}
  % turn on hanging indent if param 1 is 1
  \ifodd #1 \everypar{\setlength{\hangindent}{\cslhangindent}}\ignorespaces\fi
  % set entry spacing
  \ifnum #2 > 0
  \setlength{\parskip}{#2\baselineskip}
  \fi
 }%
 {}
\usepackage{calc} % for calculating minipage widths
\newcommand{\CSLBlock}[1]{#1\hfill\break}
\newcommand{\CSLLeftMargin}[1]{\parbox[t]{\csllabelwidth}{#1}}
\newcommand{\CSLRightInline}[1]{\parbox[t]{\linewidth - \csllabelwidth}{#1}\break}
\newcommand{\CSLIndent}[1]{\hspace{\cslhangindent}#1}


\templatetype{pnasresearcharticle}  % Choose template

\title{Impermanence of cropland abandonment limits potential for
environmental benefits}

\author[a,1]{Christopher L. Crawford}
\author[b]{He Yin}
\author[c]{Volker C. Radeloff}
\author[a,d]{David S. Wilcove}

  \affil[a]{Princeton School of Public and International Affairs,
Princeton University, Princeton, NJ}
  \affil[b]{Department of Geography, Kent State University, Kent, OH}
  \affil[c]{SILVIS Lab, Department of Forest \& Wildlife Ecology,
University of Wisconsin - Madison, Madison, WI}
  \affil[d]{Department of Ecology \& Evolutionary Biology, Princeton
University, Princeton, NJ}


% Please give the surname of the lead author for the running footer
\leadauthor{Crawford}

% Please add here a significance statement to explain the relevance of your work
\significancestatement{Demographic, economic, and environmental changes
often result in cropland abandonment. While sometimes seen as a threat
to food security and cultural landscapes, abandonment is increasing
viewed as an opportunity for carbon sequestration and habitat
restoration. However, those environmental benefits largely depend on how
long ecosystems have to regenerate and accumulate carbon and
biodiversity following abandonment. Existing studies often rely on
single snapshots in time to estimate abandonment, and ignore
recultivation. Using annual land-cover maps, we tracked abandonment and
recultivation at eleven sites across four continents. Contrary to
previous assumptions, abandonment was short-lived and most abandoned
croplands were recultivated within three decades. Unless policymakers
take steps to reduce recultivation and provide incentives for
regeneration, abandonment will remain a missed opportunity.}


\authorcontributions{Please provide details of author contributions
here.}

\authordeclaration{Please declare any conflict of interest here.}


\correspondingauthor{\textsuperscript{1} To whom correspondence should
be addressed. E-mail:
\href{mailto:ccrawford@princeton.edu}{\nolinkurl{ccrawford@princeton.edu}}}

% Keywords are not mandatory, but authors are strongly encouraged to provide them. If provided, please include two to five keywords, separated by the pipe symbol, e.g:
 \keywords{  Agriculture |  Cropland abandonment |  Biodiversity
conservation |  Secondary succession |  Carbon
sequestration |  Land-cover mapping  } 

\begin{abstract}
Agricultural expansion is a major cause of land-use change globally, but
millions of hectares of cropland are simultaneously being abandoned as a
result of demographic, economic, and environmental changes. These
abandoned croplands could be immensely valuable for carbon sequestration
and biodiversity restoration. However, their environmental value depends
on the duration and persistence of abandonment, which is poorly known.
Here, we use 30-m annual land-cover maps to examine the duration and
persistence of abandonment at eleven sites across four continents that
experienced cropland abandonment from 1987 and 2017. We find that
abandonment is by and large fleeting, lasting on average only 14.42
years (SD = 1.52). At most sites, we project \textgreater50\% of
abandoned croplands to be recultivated within 25 years, and sites in
Eastern Europe, Russia, and the US had accelerating rates of
recultivation. If current rates of abandonment and recultivation
persist, the mean duration of abandonment at most sites will plateau
between 10 and 22.3 years by about 2040. Counter to optimistic
assumptions, most abandonment is unlikely to be permanent and will
produce little in the way of benefits for carbon sequestration or
biodiversity conservation as a result. New policies and incentives will
be needed to lengthen the period of abandonment so as to generate these
and other benefits. Until then, abandoned croplands will remain untapped
opportunities.
\end{abstract}

\dates{This manuscript was compiled on \today}
\doi{\url{www.pnas.org/cgi/doi/10.1073/pnas.XXXXXXXXXX}}

\begin{document}

% Optional adjustment to line up main text (after abstract) of first page with line numbers, when using both lineno and twocolumn options.
% You should only change this length when you've finalised the article contents.
\verticaladjustment{-2pt}

\maketitle
\thispagestyle{firststyle}
\ifthenelse{\boolean{shortarticle}}{\ifthenelse{\boolean{singlecolumn}}{\abscontentformatted}{\abscontent}}{}

% If your first paragraph (i.e. with the \dropcap) contains a list environment (quote, quotation, theorem, definition, enumerate, itemize...), the line after the list may have some extra indentation. If this is the case, add \parshape=0 to the end of the list environment.

\acknow{We thank the following researchers at the University of
Wisconsin-Madison, whose land cover maps made this analysis possible:
Amintas Brandão, Johanna Buchner, David Helmers, Benjamin G. Iuliano,
Niwaeli E. Kimambo, Katarzyna E. Lewińska, Elena Razenkova, Afag
Rizayeva, Natalia Rogova, Seth A. Spawn, \& Yanhua Xie. We are deeply
indebted to Alex Wiebe, whose statistical advice significantly improved
our models of abandonment decay and recultivation. We also thank the
Drongos research group for invaluable advice and companionship. This
work was supported by the High Meadows Foundation and performed using
Princeton Research Computing resources at Princeton University.}

\hypertarget{abstract}{%
\section{Abstract}\label{abstract}}

Agricultural expansion is a major cause of land-use change globally, but
millions of hectares of cropland are simultaneously being abandoned as a
result of demographic, economic, and environmental changes. These
abandoned croplands could be immensely valuable for carbon sequestration
and biodiversity restoration. However, their environmental value depends
on the duration and persistence of abandonment, which is poorly known.
Here, we use 30-m annual land-cover maps to examine the duration and
persistence of abandonment at eleven sites across four continents that
experienced cropland abandonment from 1987 and 2017. We find that
abandonment is by and large fleeting, lasting on average only 14.42
years (SD = 1.52). At most sites, we project \textgreater50\% of
abandoned croplands to be recultivated within 25 years, and sites in
Eastern Europe, Russia, and the US had accelerating rates of
recultivation. If current rates of abandonment and recultivation
persist, the mean duration of abandonment at most sites will plateau
between 10 and 22.3 years by about 2040. Counter to optimistic
assumptions, most abandonment is unlikely to be permanent and will
produce little in the way of benefits for carbon sequestration or
biodiversity conservation as a result. New policies and incentives will
be needed to lengthen the period of abandonment so as to generate these
and other benefits. Until then, abandoned croplands will remain untapped
opportunities.

\hypertarget{significance-statement}{%
\section{Significance Statement}\label{significance-statement}}

Demographic, economic, and environmental changes often result in
cropland abandonment. While sometimes seen as a threat to food security
and cultural landscapes, abandonment is increasing viewed as an
opportunity for carbon sequestration and habitat restoration. However,
those environmental benefits largely depend on how long ecosystems have
to regenerate and accumulate carbon and biodiversity following
abandonment. Existing studies often rely on single snapshots in time to
estimate abandonment, and ignore recultivation. Using annual land-cover
maps, we tracked abandonment and recultivation at eleven sites across
four continents. Contrary to previous assumptions, abandonment was
short-lived and most abandoned croplands were recultivated within three
decades. Unless policymakers take steps to reduce recultivation and
provide incentives for regeneration, abandonment will remain a missed
opportunity.

\hypertarget{introduction}{%
\section{Introduction}\label{introduction}}

Human populations are in flux around the world, as people seek new
economic opportunities in cities and flee changing environments and
conflicts (1). Coupled with environmental degradation and changing
agricultural technologies, urbanization and rural outmigration have
contributed to a decline in the economic viability of some agricultural
lands, resulting in a growing global trend of agricultural abandonment
(2, 3). Given increasing competition for land, these abandoned
agricultural lands are highly sought after for diverse goals such as
biofuel production (4), carbon sequestration (5, 6), or recultivation
(7). To conservationists, these lands represent a potentially large
opportunity for biodiversity recovery as former croplands regenerate
into native vegetation (8--10).

Millions of hectares of agricultural lands have been abandoned over the
last half century (11), with abandonment predicted to continue in many
places (1, 12), but for how long and to what end is unclear.
Understanding the environmental effects of agricultural abandonment
requires not only detailed information on where and when abandonment
takes place, but critically, what happens to croplands after they are
abandoned. In order for abandoned lands to yield environmental benefits,
they must stay abandoned long enough for appreciable gains to be made in
both plant biomass and the species that make up intact ecological
communities, which can take many decades to approach the levels of
intact reference ecosystems, with species recolonizing at different
successional stages (13--19). Knowing how long abandonment persists is
critical to understanding its potential to help mitigate the ongoing
climate and biodiversity crises.

Until recently, it has been difficult to gather detailed information on
abandonment. Many estimates of abandonment are inferred by aggregating
regional estimates of cultivation, such as from country-level FAO data
on cultivated areas (15, 20). When derived from satellite imagery,
abandonment is most frequently estimated by simply taking the difference
in cultivated area between land cover maps acquired at two distinct
points in time (e.g.~1992 and 2015 (4)). Annual land-cover time series
are becoming more common, but are typically limited to a short timespan
(e.g.~2-12 years (21--23); though see (24)), and most analyses are
limited to a single region only. These approaches all lack the spatial
and temporal detail needed to understand long-term outcomes for
croplands at field level, and by failing to capture the dynamic patterns
of abandonment and recultivation, may significantly overestimate
abandonment (25).

Empirical evidence for secondary forests in the Neotropics shows that
forest regeneration is often short-lived (26--29), according to the
mapping of woody regrowth in satellite imagery. In the Brazilian Amazon,
half of secondary forests were recleared in as little as 5 to 8 years
(6, 27, 30), and in southern Costa Rica in \(\leq20\) years (26).
However, estimates based on woody vegetation growth miss abandonment in
earlier stages of regeneration and in biomes where succession does not
lead to woody vegetation (25).

Recent advances in remote sensing have made it possible to produce maps
of cropland abandonment at both high spatial and temporal resolution.
Yin et al. (25) used a trajectory-based approach to produce accurate
maps of annual cropland abandonment at 30-m resolution without relying
on woody revegetation, allowing for a direct focus on abandonment and
analyses of both forest and non-forest biomes. Here, we utilize these
new annual time-series of cropland abandonment (25) to investigate the
duration and persistence of abandonment at eleven sites across four
continents, including sites in Brazil, the United States, Eastern
Europe, Russia, the Middle East, and China (Figure
\ref{fig:mean-abn-duration}).

We address three questions: How long does abandoned cropland stay
abandoned (i.e.~duration of abandonment), and how does this vary
geographically? How quickly is abandoned land recultivated
(i.e.~persistence of abandonment), and do recultivation (or ``decay'')
rates vary through time? Finally, if abandonment and recultivation
continue at current rates, what will ultimately be the mean age of
abandoned croplands?

By making use of new, more accurate and finer resolution abandonment
maps for eleven sites across the globe, and by covering a longer period
and broader set of sites than previous studies, we provide the most
detailed analysis to date of the duration and persistence of cropland
abandonment. Given that some abandonment periods are inherently limited
by the length of our time series (and may remain abandoned beyond the
three decades covered by our data), tracking recultivation rates for
each year provides a more accurate understanding of persistence.
Together, these results highlight the temporal nature of cropland
abandonment and its potential to sequester carbon and conserve
biodiversity.

\hypertarget{results}{%
\section{Results}\label{results}}

\begin{figure*}
\includegraphics{/Users/christophercrawford/Google_Drive/_Projects/abandonment_trajectories/output/plots/fig1} \caption{Observed duration of cropland
abandonment (in years) as of 2017 in our eleven study sites. Site
locations are shown in Figure \ref{fig:global-map}.}\label{fig:maps-abn-duration}
\end{figure*}

\hypertarget{abandonment-duration}{%
\subsection{Abandonment duration}\label{abandonment-duration}}

Cropland abandonment was widespread across our eleven study sites. We
detected 5.73 million ha (Mha) of abandoned croplands across our eleven
sites as of 2017 (Figure \ref{fig:mean-abn-duration}).\footnote{\emph{As
  a reminder, this is the total abandonment \textbf{as of 2017}. I
  report the area that was abandoned at least once during the time
  series below (8.32 Mha), but please let me know if you think I should
  change the order in which the numbers are reported. I could also
  update the percentages to reflect the ``total area ever abandoned''
  number.}} This corresponds to 24.15\% of the total cropland extent,
23.74 Mha (i.e.~all lands that were cultivated at some point during the
time series). Cropland abandonment at individual sites as of 2017 ranged
from 17.86\% (Nebraska) to 41.06\% (Bosnia \& Herzegovina) of the total
cropland extent (Figures \ref{fig:area-abn-by-age-class} and
\ref{fig:area-abn-panel}\footnote{Note: figure numbers above 4 will be
  updated to read ``Fig. S1,'' etc. in the final version.}), except for
Mato Grosso, where only 0.47\% was abandoned as of 2017 (see Section
\ref{mato-grosso}).

However, we also found that many of these abandoned croplands were
recultivated: 37.77\% of abandoned area on average (SD = 8.9\%; Figure
\ref{fig:recult-by-threshold}). In fact, 8.36 Mha of croplands were
abandoned at least once during our time series. Accordingly, the mean
duration of abandonment across all sites was short: 14.42 years (SD =
1.52), ranging from 12.86 years (Orenburg) to 17.7 years (Bosnia \&
Herzegovina; Figure \ref{fig:mean-abn-duration}). Abandonment duration
also varied substantially within sites, with individual site standard
deviations ranging between 6.95 (Orenburg) and 9.21 years (Mato Grosso)
(for an average of 7.78 years across all sites).

\hypertarget{modeling-abandonment-decay-recultivation}{%
\subsection{Modeling abandonment decay \&
recultivation}\label{modeling-abandonment-decay-recultivation}}

We defined a cohort of abandoned cropland as all cropland abandoned in a
given year at a given site. We modeled the proportion of each cohort
remaining abandoned as a function of time since initial abandonment (see
Section \ref{methods}, Figure \ref{fig:abn-decay-s}). This allows us to
predict how long a given abandoned cropland will persist before it is
recultivated, regardless of when it was abandoned during the time
series, and also calculate the mean recultivation (or ``decay'')
trajectory at each site (Figure \ref{fig:decay-curves-by-site}).

Recultivation occurred relatively quickly: \textgreater50\% of abandoned
croplands were projected to be recultivated within 25 years at almost
all sites, ranging between 12 years (Orenburg) and 24 years (Wisconsin).
The primary exception was Shaanxi/Shanxi (China), where abandonment
persisted. Our models predicted a half-life of 48 years (here defined as
the time required for half of the croplands abandoned in a given year to
be recultivated), and 137 years for complete turnover (here defined as
the time required for all abandonment in a given cohort to be
recultivated) (Figure \ref{fig:decay-curves-by-site}). These mean decay
trajectories also display wide variation in the proportion of abandoned
croplands that persist beyond 20 years: as few as 7.46\% in Orenburg, up
to 70.9\% in Shaanxi/Shanxi (China).


By modeling recultivation for each cohort of abandonment, we are also
able to investigate whether recultivation of abandoned cropland is
accelerating. Most sites showed a negative trend in the time required
for recultivation, indicating that abandoned croplands were being
recultivated more quickly in recent years (Figure
\ref{fig:decay-rate-of-change}). Based on a linear regression on the
half-life (the time required for 50\% recultivation), over half of our
sites had rates of change that were significantly different from zero
and negative, corresponding to recultivation accelerating through time:
Bosnia \& Herzegovina, Volgograd (Russia), Wisconsin (USA), Nebraska
(USA), Orenburg (Russia) / Uralsk (Kazakhstan), and Iraq (Figure
\ref{fig:decay-rate-of-change}). The remaining sites showed trends that
were not significantly different from zero.

\hypertarget{extrapolating-abandonment-and-recultivation}{%
\subsection{Extrapolating abandonment and
recultivation}\label{extrapolating-abandonment-and-recultivation}}

Assuming that abandonment rates and recultivation rates remain constant
at each site, our extrapolation suggests that the impermanence we
observe is not an artifact of our time series being only 30 years long.
We expect the mean duration of abandonment to plateau at 20 years or
less at most sites (Figure \ref{fig:extrapolation-combo}), with the
notable exception of Shaanxi/Shanxi (China), which is projected to
plateau at a longer mean duration of 46 years. The time required to
plateau depends on the time required for complete turnover, ranging
between 21-61 years for all sites except Mato Grosso (110 years) and
Shaanxi/Shanxi (China) (137 years).

Furthermore, only a relatively small proportion of abandoned croplands
are projected to persist beyond 30 years (less than 27\% at most sites;
Figure \ref{fig:extrapolation-area-by-age}). Even at Shaanxi/Shanxi
(China), the site projected to have the most durable abandonment, only
40\% of abandoned land will remain so for longer than 50 years.


\begin{figure}
\includegraphics{/Users/christophercrawford/Google_Drive/_Projects/abandonment_trajectories/output/plots//fig2} 
\caption{Mean length of time abandoned (years) throughout our time series across study sites. Because abandonment and
recultivation can occur multiple times during our time series, we show both the mean abandonment duration calculated across all periods of abandonment (in red) and the maximum length of abandonment (in blue).)}
\label{fig:mean-abn-duration}
\end{figure}

\label{caption-decay-curves-by-site} Mean decay trajectories for each
site, based on a linear model predicting the proportion of abandoned
land remaining abandoned as a function of time (including a linear and a
logarithmic term of time). The function describing each site's mean
trajectory is calculated by taking the mean of each time coefficient
across all cohorts of abandonment at each site.

\begin{figure*}
\includegraphics{/Users/christophercrawford/Google_Drive/_Projects/abandonment_trajectories/output/plots//fig3} 
\caption{\ref{caption-decay-curves-by-site}}
\label{fig:decay-curves-by-site}
\end{figure*}

\hypertarget{discussion}{%
\section{Discussion}\label{discussion}}

Using a new detailed and annual land-cover time series, we uncovered
significant amounts of abandonment from 1987 to 2017 in eleven sites
across the globe: 24.15\% of all lands that were cultivated at least
once during our time series were abandoned as of 2017. However, we also
found substantial levels of recultivation, so that on average, 37\% of
abandoned croplands were recultivated by 2017, and abandonment lasted on
average only 15 years. Furthermore, some sites had accelerating rates of
recultivation, and even the sites with the most persistent abandonment
are unlikely to remain uncultivated for \(\geq50\) years. Our modeled
decay rates portray a more dynamic process, where abandonment is rarely
an endpoint but rather part of a cycle of turnover on decadal timescales
(20).

Furthermore, by using an annual time series and tracking cohorts of
abandoned croplands based on the year of abandonment, we developed a
much more accurate picture of abandonment persistence. For example,
while Shaanxi/Shanxi (China) had among the shortest abandonment
durations, current recultivation trajectories indicate more persistent
abandonment than we can observe during the time series, given an
increase in the rate of abandonment coupled with a decrease in
recultivation rates over time To the contrary, Bosnia \& Herzegovina,
which had the longest mean abandonment duration, experienced
accelerating recultivation in recent years, challenging future
persistence.

The abandonment and impermanence that we observe generally matches that
observed in the few other case studies we found (6). To the best of our
knowledge, the only other study that uses an annual time series to
investigate agricultural abandonment (in a grassland region of northern
Kazakhstan between 1991-2017; (24)), observed abandonment of about 40\%
of cultivated areas, and the subsequent recultivation of about 20\% of
that abandonment. This recultivation rate was similar to our most
persistent site (Shaanxi/Shanxi {[}China{]}), but lower than both our
average across sites (37.77\%), and our closest sites: Orenburg and
Volgograd, Russia, which had recultivation rates of 43.34\% and 45.52\%
respectively (Figure \ref{fig:recult-by-threshold}).

Our projected half-lives of 12-24 years at most sites were similar to
those in Costa Rica (20 years (26)), but slightly longer than in other
parts of the Neotropics (6, 27, 28). Secondary forests were recleared
more quickly in the Brazilian Amazon (50\% within 5 to 8 years), where
80\% of secondary forests were \(\leq20\) years old (6, 27, 28). Across
the tropics, only 33\% of forests regenerating after recent
deforestation were \(\geq10\) years (31). However, these differences may
be the result of 1) a time delay between abandonment and the regrowth of
secondary woody vegetation that can be detected by satellites, 2) our
exclusion of abandonment less than five years, or 3) our models of
recultivation for each cohort, which eliminate the influence of the time
series length, and lengthen abandonment estimates.

\hypertarget{implications-for-biodiversity-recovery-and-carbon-sequestration}{%
\subsection{Implications for biodiversity recovery and carbon
sequestration}\label{implications-for-biodiversity-recovery-and-carbon-sequestration}}

Our results show that most cropland abandonment is unlikely to be
permanent, contrary to optimistic assumptions (4, 32). Estimates of
abandonment based on short time series or two points in time (e.g.~83
Mha estimated by Næss et al. (4)) likely overestimate the amount of
durable abandonment, not only because of a failure to exclude short-term
fallow periods, but also, as our analysis shows, as a result of high
rates of recultivation. When estimating abandonment at our sites based
on cultivation in two points in time (1987 and 2017), we find 5 Mha of
``abandonment,'' a 12.74\% underestimate compared to the 5.73 Mha of
abandonment as of 2017 identified using our full annual time series.
This corresponds to site-level estimates that ranged between 39.75\%
less (Goiás, Brazil) and 83.29\% more (Mato Grosso, Brazil) area
identified using the full time series. Furthermore, this two-year method
identifies different areas of abandonment from those identified using
our full annual time series, with spatial agreement of between 28.08\%
(Mato Grosso, Brazil) and 66.3\% (Bosnia \& Herzegovina) (see Table
\ref{tab:twoyr-diff-table}, Section \ref{twoyr-vs-annual}).
Recultivation may dramatically limit the scope for abandoned croplands
to play a major role in carbon sequestration or the regeneration of
biodiversity.

There is substantial variability in how quickly and how completely
ecosystems recover following disturbances and abandonment (33--35).
Natural regeneration can be a viable ecosystem restoration strategy
under the right conditions, especially after low-intensity disturbances
(e.g.~selective logging) and when close to pristine ecosystems,
resulting in secondary growth that approaches the habitat and carbon
values of pristine ecosystems (17, 34, 35).

However, even under optimal conditions, recovery requires time. In most
cases, it takes multiple decades to recover species richness to values
close to those in reference systems (8, 14--16, 36). Rarer,
forest-adapted, and old-growth dependent species return even more
slowly, and the recovery towards old-growth ecosystems in terms of
community composition, species similarity, and vegetation structure can
take much longer than simply the recovery of total species richness or
abundance, which can be dominated by widespread generalist species
(34--37).

Tropical ecosystems are among the most speciose in the world. In some
cases they can regenerate quickly, but never on a time scale as short as
the abandonment we observe here. Neotropical chronosequences show that
lowland tropical forests recover quickly in terms of tree species
richness (reaching 80\% of old-growth levels after 20 years, 90\% after
31 years), but much more slowly in terms of tree species composition
(34\% of old-growth levels after 20 years, requiring 487 years to reach
90\%) (13). These sites retained relatively high forest cover (76\% on
average), however, and recovery is likely to be slower in less-forested
ecosystems (13).

Recovery can be rapid for some vertebrate groups (17), but recovery
typically takes a long time for most. Across tropical forests,
amphibian, bird, mammal, and reptile species richness largely recovers
within 40 years, but species compositional similarity takes much longer
to recover (if at all), particularly for late-successional species,
insectivores, and forest specialists (14). No vertebrate groups reach
similarity to reference old-growth forests, even in the oldest secondary
forests (30-65 years).

Grassland ecosystems can sometimes recover more quickly following
disturbances than forests (38), but not often (15, 16). On average,
secondary grasslands distributed around the world and between 1-251
years old contained only 53-76\% of the plant species richness of
old-growth grasslands (16); even after full recovery of species richness
(with minimum estimated recovery times of \textgreater100 years),
compositional similarity to old-growth grasslands remained low (43\%).
Similarly, after quick initial gains in biodiversity following
abandonment in Minnesota (USA) (averaging over 1/3 diversity after just
one year), both biodiversity and productivity increased slowly over
time, reaching only 73\% of the diversity and 53\% of the productivity
levels of a reference ecosystem after 91 years (15). Eurasian
grasslands, where widespread abandonment has occurred, do not fully
recover in either plant species richness or community composition after
24 years (39), nor bird species richness and diversity after 18 years
(40).

Abandoned areas may also only achieve a small fraction of their carbon
storage potential if abandonment last only a couple of decades. Despite
relatively quick accumulation of carbon in aboveground biomass over the
first few decades of regeneration, it can take between 50-100 years for
secondary forests to achieve similar levels of biomass as old-growth
forests (19, 41). For example, Neotropical aboveground forest biomass
reaches 90\% of old-growth forest biomass after a median of 66 years,
but \textless50\% biomass after 20 years. Furthermore, like biodiversity
recovery, carbon accumulation estimates vary by biome and by prior land
use. Based on estimates of potential aboveground carbon accumulation
rates during the first 30 years of natural regeneration in different
biomes (18), we predict that the abandoned cropland areas that we
studied could accumulate 30-70 Mg C in aboveground biomass per ha over
20 years, or 18\%-45\% of the 110-250 Mg C per ha that could be
accumulated in aboveground biomass over 100 years (18). Much of the
abandonment we observe occurred in grassy biomes, where soil carbon
sequestration can also be substantial. However, soil carbon may
accumulate even more slowly, taking a century or longer to return to
reference levels (42).

In addition to the short duration of abandonment, just abandoning
croplands may be insufficient to achieve carbon or biodiversity goals.
The degradation and slow recovery of soils, a lack of nearby source
populations, a lack of natural disturbance regimes, and climate change
all pose problems (34, 43, 44). While natural regeneration is typically
cheaper and sometimes more successful than active restoration under the
right conditions, more heavily altered systems typically require more
active approaches and longer recovery times (17, 36, 45). Grassland
regeneration is particularly challenging, because many grasslands
require natural disturbance regimes (e.g.~grazing or fire), which may be
absent following abandonment (16). Without such disturbances, grassland
biodiversity may be lost if low-intensity farmlands are abandoned and
replaced by woody vegetation, particularly in historically unforested
ecosystems (2, 16, 46, 47).

\hypertarget{conclusions}{%
\subsection{Conclusions}\label{conclusions}}

We found strong evidence that abandoned croplands did not remain
uncultivated for long periods of time at eleven sites across the globe.
Without the development of policies and incentives to discourage
recultivation, abandoned areas are unlikely to provide meaningful
biodiversity and carbon benefits. Our results reinforce growing calls
for enhanced monitoring of regeneration (48), and the development of a
stronger policy framework for managing such lands (8).

In addition to the biophysical challenges outlined above, there are a
range of socioeconomic and political barriers hindering habitat
regeneration in abandoned croplands. These include policies that
obligate farmers to cultivate land, a lack of incentives to protect and
foster regenerating habitats, and perhaps most importantly, negative
cultural perceptions of ``abandonment'' and the ``messy'' landscapes
that result (8). Agricultural abandonment reflects broad societal
changes, and the pain associated with the loss of certain types of
landscapes and rural ways of life is non-trivial (2, 7, 11, 20).
Behavioral research on farmer decisions to recultivate abandoned
croplands highlights the importance of addressing factors such as
corruption, political and institutional support for agriculture, and
demographics, alongside biophysical and environmental conditions, when
managing abandonment (49).

In order for abandoned croplands to provide environmental benefits
through habitat regeneration or carbon sequestration, they must persist
for longer. This could involve including abandoned fields in protected
areas, incorporating natural regeneration into ecosystem service schemes
to allow landowners to benefit economically, or taking steps to support
sustainable long-term cultivation and reduce turnover among fields that
have previously been part of long-term fallowing cycles. The relative
durability of abandonment at Shaanxi/Shanxi (China) may be due to
large-scale reforestation programs implemented over the last 25 years,
which is encouraging, even though there continues to be a need to
improve the biodiversity outcomes of such programs (50). Such policies
should be developed alongside local communities in order to best manage
the trade-offs between biodiversity, carbon storage, and livelihoods.

Our results make one thing clear: if cropland abandonment continues to
be as short-lived as we show here, the large potential benefits of
regenerating habitats to both store carbon and sustain biodiversity will
remain an untapped opportunity.

\hypertarget{methods}{%
\section{Materials and Methods}\label{methods}}

\hypertarget{abandonment-maps}{%
\subsection{Abandonment maps}\label{abandonment-maps}}

We use annual land cover maps with 30-m resolution from 1987-2017 (25),
derived from publicly available Landsat satellite imagery, mapping four
land cover classes: 1) cropland, 2) herbaceous vegetation
(e.g.~grassland), 3) woody vegetation (e.g.~forests), and 4)
non-vegetation (e.g.~water, urban, or barren land). Our eleven sites
were mapped with high accuracy (average overall accuracy \(85\pm4\)\%)
and provide broad coverage of different continents and ecosystems
(Figure \ref{fig:mean-abn-duration}). We focused exclusively on cropland
abandonment, because pasture abandonment is very difficult to discern
from satellite imagery and is not captured in our data. Yin et al. (25)
selected sites where recent abandonment was documented and likely due to
socioeconomic, political, or environmental factors. Thus, while our
results are likely representative of those areas that have recently
experienced abandonment, they are not a representative sample for the
globe, and overestimate the overall prevalence of abandonment.

\hypertarget{defining-abandonment}{%
\subsection{Defining abandonment}\label{defining-abandonment}}

Differentiating abandonment from short-term fallowing or crop rotations
is difficult because agricultural practices can vary widely by region,
and studies use many different definitions (25). Here, we define
``abandonment'' as cropland that is no longer under active cultivation
for at least five years, and is left free of direct human influence,
therefore excluding, for example, conversion to urban land use. To
exclude short-term fallowing, we define croplands as abandoned when they
remain uncultivated for five or more consecutive years, following FAO
(51). Recognizing that longer abandonment thresholds may be more
appropriate in certain contexts, we performed a sensitivity analysis by
varying our abandonment definition (Section \ref{abn-thresholds}) and
found that, as expected, longer definitions resulted in less abandonment
over all, longer average abandonment durations (Figure
\ref{fig:abn-thresholds-mean-duration}), and lower recultivation rates
(Figure \ref{fig:recult-by-threshold}). However, even when only
considering abandonment longer than 10 years, we still observed between
11.61\% and 30.37\% recultivation across our sites (at Shaanxi/Shanxi
{[}China{]} and Volgograd {[}Russia{]} respectively), enough to
substantially curtail regeneration and any associated benefits.

\hypertarget{data-processing}{%
\subsection{Data processing}\label{data-processing}}

We processed and analyzed abandonment map data in RStudio version
1.4.1717 (52), using R version 4.1.0 (2021-05-18), primarily with the
\texttt{raster} (53), \texttt{data.table} (54), and \texttt{tidyverse}
(55) packages.

We identified periods of cropland abandonment by tracking each pixel's
land cover through time and looking for land-cover changes that
indicated transitions between agricultural activity and abandonment. We
classified a pixel as ``abandoned'' anytime it transitioned from
cropland to either herbaceous or woody vegetation (collectively referred
to as ``non-cropland'') and subsequently remained classified as
non-cropland for five or more consecutive years. We considered an
abandoned pixel to be ``recultivated'' when it transitioned from
abandoned to cropland. Pixels that transitioned from cropland to the
non-vegetation class were not considered ``abandoned,'' and were
excluded from our analysis. Non vegetated land consisted of
\textless10\% of total site area at all sites except Shaanxi (12.7\%)
and Iraq (52.8\%), and remained stable or declined over time at all
eleven sites.

We implemented five- and eight-year moving window temporal filters to
smooth land-cover trajectories and address land-cover changes that are
temporally unlikely (Section \ref{methods-si}). Together with our
five-year abandonment threshold, these temporal filters address very
short-term misclassifications that might otherwise look like
recultivation.

\hypertarget{calculating-abandonment-duration}{%
\subsection{Calculating abandonment
duration}\label{calculating-abandonment-duration}}

We calculated abandonment duration as the number of years that elapse
between the initial transition from cropland to non-cropland, and either
recultivation or the end of the time series. Our abandonment definition
implies a minimum abandonment duration of five years. Because a pixel
may be abandoned and recultivated multiple times throughout the time
series, we calculated the mean abandonment duration in two ways: 1)
across all periods of abandonment, and 2) across only the longest period
of abandonment experienced by each pixel (Figure
\ref{fig:mean-abn-duration}).

\hypertarget{modeling-abandonment-decay}{%
\subsection{Modeling abandonment
decay}\label{modeling-abandonment-decay}}

Because some abandonment periods are limited by the length of the time
series, we modeled recultivation of abandoned croplands as a function of
time since initial abandonment. We tracked recultivation (``decay'') by
calculating the proportion of each cohort of pixels abandoned in a given
year that remain abandoned in each year following abandonment. We
parameterized linear models predicting the proportion of abandoned
cropland in each cohort remaining abandoned as a function of time since
initial abandonment. We tested a range of model specifications,
including linear and log transformations of both \emph{proportion} and
\emph{time}, and multiple \emph{time} predictor terms. Importantly, we
included \emph{site} and \emph{cohort} level fixed effects, fitting
unique coefficients for each cohort at each site.

We selected the highest performing model based on Akaike Information
Criterion values (AIC; Figure \ref{fig:AIC}). For cohorts of abandonment
initially abandoned in years \(y = 1988 ... 2013\), our model predicted
the proportion \(p\) of each cohort \(y\) remaining abandoned as a
function of time \(t\) (i.e.~the number of years following initial
abandonment), with one log-transformed and one linear term of time
(Equation \ref{eq:mod-spec}).

\begin{equation}
p_{y} = 1 + \beta_{1,y} log(t + 1) + \beta_{2,y} t (\#eq:mod-spec)
\end{equation}

Where \(\beta_{1,y}\) and \(\beta_{2,y}\) represent the regression
coefficients on the log and linear terms of time \(t\), respectively,
for cohort \(y\). Model assumptions were tested through visual
inspection of diagnostic plots (Figures \ref{fig:diag-resid-fitted} and
\ref{fig:diag-qq}). See full details in Section \ref{decay-models-si}.
We calculated each site's mean recultivation trend by taking the mean of
the model coefficients across all cohorts (Figure
\ref{fig:mean-abn-duration}). Modeled decay trajectories are shown for
one example site, Shaanxi/Shanxi (China), in Figure
\ref{fig:abn-decay-s}, and model coefficients for all sites in Figure
\ref{fig:decay-mod-coef}.

To estimate changes in persistence over time, we calculated the time
required for half of a given cohort to be recultivated based on the
modeled recultivation trajectory of each cohort. We parameterized a
linear model on this value for each cohort to identify temporal changes
in recultivation patterns at each site (Section
\ref{section-methods-rate-of-change}). Trends were considered
statistically significant when the 95\% confidence interval for model
coefficients did not include zero (Figure
\ref{fig:decay-rate-of-change}).

\ref{caption-abn-decay-s} Decay of abandoned land in Shaanxi/Shanxi
(China), showing the proportion of each cohort of abandoned land
(i.e.~all pixels abandoned in a given year) remaining abandoned over
time. Points represent actual observations by cohort, dashed green lines
represent linear model predictions for each cohort as a function of time
(including a linear and lagarithmic term of time), and the solid purple
line represents the mean trend across all cohorts (calculated by taking
the mean of each time coefficient value across all cohorts). For similar
results for our other sites, see Figures \ref{fig:decay-model-grid} and
\ref{fig:decay-model-indiv-site-b}-\ref{fig:decay-model-indiv-site-w}.

\begin{figure*}
\includegraphics{/Users/christophercrawford/Google_Drive/_Projects/abandonment_trajectories/output/plots/fig4} 
\caption{\ref{caption-abn-decay-s}}
\label{fig:abn-decay-s}
\end{figure*}

\hypertarget{extrapolation}{%
\subsubsection{Extrapolation}\label{extrapolation}}

Based on our modeled decay rates, we forecast abandonment and
recultivation in the future, based on two assumptions: (1) a constant
amount of cropland is newly abandoned each year, based on the mean area
of cropland abandoned each year at each site (Figure
\ref{fig:turnover-grid}); (2) all abandoned croplands are recultivated
at the mean predicted rate for each site (Figure
\ref{fig:decay-curves-by-site}). Total abandonment in our extrapolation
remained below the total cropland extent at each site. We also
considered an alternative assumption in which annual abandonment
linearly declined to 0 between 2017 and 2050, but found similarly short
mean abandonment durations (below 37 years at all sites by 2050; see
Figures \ref{fig:extrapolation2-combo} and
\ref{fig:extrapolation2-area-by-age}). While mean age of abandoned land
increased through time after 2017 (and most dramatically after 2050), it
remained below 37 years at all sites by 2050. Increases in mean
abandonment duration were offset by recultivation, and total area
abandoned declined quickly after 2020 at most sites. See details in
Section \ref{section-extrapolation-si}.

\hypertarget{data-availability-statement}{%
\section{Data Availability
Statement}\label{data-availability-statement}}

Code to replicate these analyses is available on GitHub at
\url{https://github.com/chriscra/abandonment_trajectories}, and is
archived at \href{https://doi.org/}{Zenodo (TBD)}.

\hypertarget{acknowledgements}{%
\section{Acknowledgements}\label{acknowledgements}}

We thank the following researchers at the University of
Wisconsin-Madison, whose land cover maps made this analysis possible:
Amintas Brandão, Johanna Buchner, David Helmers, Benjamin G. Iuliano,
Niwaeli E. Kimambo, Katarzyna E. Lewińska, Elena Razenkova, Afag
Rizayeva, Natalia Rogova, Seth A. Spawn, \& Yanhua Xie. We are deeply
indebted to Alex Wiebe, whose statistical advice significantly improved
our models of abandonment decay and recultivation. We also thank the
Drongos research group for invaluable advice and companionship. This
work was supported by the High Meadows Foundation and performed using
Princeton Research Computing resources at Princeton University.

\hypertarget{supporting-information}{%
\section{Supporting Information}\label{supporting-information}}

A version of this manuscript that includes the Supporting Information
\href{https://drive.google.com/file/d/1Y_tnucb_Zia-4qE4-vc02s5hFuMHnIZ9/view?usp=sharing}{can
be found here}.

\newpage

\hypertarget{references}{%
\section{References}\label{references}}

\hypertarget{refs}{}
\begin{CSLReferences}{0}{0}
\leavevmode\hypertarget{ref-Sanderson2018}{}%
\CSLLeftMargin{1. }
\CSLRightInline{Sanderson EW, Walston J, Robinson JG (2018) {From
Bottleneck to Breakthrough: Urbanization and the Future of Biodiversity
Conservation}. \emph{BioScience} 68(6):412--426.}

\leavevmode\hypertarget{ref-ReyBenayas2007}{}%
\CSLLeftMargin{2. }
\CSLRightInline{Rey Benayas J, Martins A, Nicolau JM, Schulz J (2007)
{Abandonment of agricultural land: an overview of drivers and
consequences.} \emph{CAB Reviews: Perspectives in Agriculture,
Veterinary Science, Nutrition and Natural Resources} 2(057).
doi:\href{https://doi.org/10.1079/PAVSNNR20072057}{10.1079/PAVSNNR20072057}.}

\leavevmode\hypertarget{ref-Prishchepov2021}{}%
\CSLLeftMargin{3. }
\CSLRightInline{Prishchepov AV, Schierhorn F, Löw F (2021) {Unraveling
the Diversity of Trajectories and Drivers of Global Agricultural Land
Abandonment}. \emph{Land} 10(2):97.}

\leavevmode\hypertarget{ref-Naess2021}{}%
\CSLLeftMargin{4. }
\CSLRightInline{Næss JS, Cavalett O, Cherubini F (2021) {The
land--energy--water nexus of global bioenergy potentials from abandoned
cropland}. \emph{Nature Sustainability}.
doi:\href{https://doi.org/10.1038/s41893-020-00680-5}{10.1038/s41893-020-00680-5}.}

\leavevmode\hypertarget{ref-Griscom2017}{}%
\CSLLeftMargin{5. }
\CSLRightInline{Griscom BW, et al. (2017) {Natural climate solutions}.
\emph{Proceedings of the National Academy of Sciences}
114(44):11645--11650.}

\leavevmode\hypertarget{ref-Chazdon2016a}{}%
\CSLLeftMargin{6. }
\CSLRightInline{Chazdon RL, et al. (2016) {Carbon sequestration
potential of second-growth forest regeneration in the Latin American
tropics}. \emph{Science Advances} 2(5):e1501639.}

\leavevmode\hypertarget{ref-Meyfroidt2016}{}%
\CSLLeftMargin{7. }
\CSLRightInline{Meyfroidt P, Schierhorn F, Prishchepov AV, Müller D,
Kuemmerle T (2016) {Drivers, constraints and trade-offs associated with
recultivating abandoned cropland in Russia, Ukraine and Kazakhstan}.
\emph{Global Environmental Change} 37:1--15.}

\leavevmode\hypertarget{ref-Chazdon2020}{}%
\CSLLeftMargin{8. }
\CSLRightInline{Chazdon RL, et al. (2020) {Fostering natural forest
regeneration on former agricultural land through economic and policy
interventions}. \emph{Environmental Research Letters} 15(4):043002.}

\leavevmode\hypertarget{ref-Xie2019}{}%
\CSLLeftMargin{9. }
\CSLRightInline{Xie Z, et al. (2019) {Conservation opportunities on
uncontested lands}. \emph{Nature Sustainability}.
doi:\href{https://doi.org/10.1038/s41893-019-0433-9}{10.1038/s41893-019-0433-9}.}

\leavevmode\hypertarget{ref-Navarro2012}{}%
\CSLLeftMargin{10. }
\CSLRightInline{Navarro LM, Pereira HM (2012) {Rewilding abandoned
landscapes in Europe}. \emph{Ecosystems} 15:900--912.}

\leavevmode\hypertarget{ref-Li2017}{}%
\CSLLeftMargin{11. }
\CSLRightInline{Li S, Li X (2017) {Global understanding of farmland
abandonment: A review and prospects}. \emph{Journal of Geographical
Sciences} 27(9):1123--1150.}

\leavevmode\hypertarget{ref-Popp2017}{}%
\CSLLeftMargin{12. }
\CSLRightInline{Popp A, et al. (2017) {Land-use futures in the shared
socio-economic pathways}. \emph{Global Environmental Change}
42:331--345.}

\leavevmode\hypertarget{ref-Rozendaal2019}{}%
\CSLLeftMargin{13. }
\CSLRightInline{Rozendaal DMA, et al. (2019) {Biodiversity recovery of
Neotropical secondary forests}. \emph{Science Advances} 5(3):eaau3114.}

\leavevmode\hypertarget{ref-Acevedo-Charry2019}{}%
\CSLLeftMargin{14. }
\CSLRightInline{Acevedo‐Charry O, Aide TM (2019) {Recovery of amphibian,
reptile, bird and mammal diversity during secondary forest succession in
the tropics}. \emph{Oikos} 128(8):1065--1078.}

\leavevmode\hypertarget{ref-Isbell2019}{}%
\CSLLeftMargin{15. }
\CSLRightInline{Isbell F, Tilman D, Reich PB, Clark AT (2019) {Deficits
of biodiversity and productivity linger a century after agricultural
abandonment}. \emph{Nature Ecology {\&} Evolution} 3(11):1533--1538.}

\leavevmode\hypertarget{ref-Nerlekar2020}{}%
\CSLLeftMargin{16. }
\CSLRightInline{Nerlekar AN, Veldman JW (2020) {High plant diversity and
slow assembly of old-growth grasslands}. \emph{Proceedings of the
National Academy of Sciences} 117(31):18550--18556.}

\leavevmode\hypertarget{ref-Gilroy2014}{}%
\CSLLeftMargin{17. }
\CSLRightInline{Gilroy JJ, et al. (2014) {Cheap carbon and biodiversity
co-benefits from forest regeneration in a hotspot of endemism}.
\emph{Nature Climate Change} 4(6):503--507.}

\leavevmode\hypertarget{ref-Cook-Patton2020}{}%
\CSLLeftMargin{18. }
\CSLRightInline{Cook-Patton SC, et al. (2020) {Mapping carbon
accumulation potential from global natural forest regrowth}.
\emph{Nature} 585(7826):545--550.}

\leavevmode\hypertarget{ref-Poorter2016}{}%
\CSLLeftMargin{19. }
\CSLRightInline{Poorter L, et al. (2016) {Biomass resilience of
Neotropical secondary forests}. \emph{Nature} 530(7589):211--214.}

\leavevmode\hypertarget{ref-Munroe2013}{}%
\CSLLeftMargin{20. }
\CSLRightInline{Munroe DK, Berkel DB van, Verburg PH, Olson JL (2013)
{Alternative trajectories of land abandonment: Causes, consequences and
research challenges}. \emph{Current Opinion in Environmental
Sustainability} 5(5):471--476.}

\leavevmode\hypertarget{ref-Alcantara2013}{}%
\CSLLeftMargin{21. }
\CSLRightInline{Alcantara C, et al. (2013) {Mapping the extent of
abandoned farmland in Central and Eastern Europe using MODIS time series
satellite data}. \emph{Environmental Research Letters} 8(3).
doi:\href{https://doi.org/10.1088/1748-9326/8/3/035035}{10.1088/1748-9326/8/3/035035}.}

\leavevmode\hypertarget{ref-Estel2015}{}%
\CSLLeftMargin{22. }
\CSLRightInline{Estel S, et al. (2015) {Mapping farmland abandonment and
recultivation across Europe using MODIS NDVI time series}. \emph{Remote
Sensing of Environment} 163:312--325.}

\leavevmode\hypertarget{ref-Lark2015}{}%
\CSLLeftMargin{23. }
\CSLRightInline{Lark TJ, Salmon JM, Gibbs HK (2015) {Cropland expansion
outpaces agricultural and biofuel policies in the United States}.
\emph{Environmental Research Letters} 10(4):044003.}

\leavevmode\hypertarget{ref-Dara2018}{}%
\CSLLeftMargin{24. }
\CSLRightInline{Dara A, et al. (2018) {Mapping the timing of cropland
abandonment and recultivation in northern Kazakhstan using annual
Landsat time series}. \emph{Remote Sensing of Environment} 213:49--60.}

\leavevmode\hypertarget{ref-Yin2020}{}%
\CSLLeftMargin{25. }
\CSLRightInline{Yin H, et al. (2020) {Monitoring cropland abandonment
with Landsat time series}. \emph{Remote Sensing of Environment}
246:111873.}

\leavevmode\hypertarget{ref-Reid2019}{}%
\CSLLeftMargin{26. }
\CSLRightInline{Reid JL, Fagan ME, Lucas J, Slaughter J, Zahawi RA
(2019) {The ephemerality of secondary forests in southern Costa Rica}.
\emph{Conservation Letters} 12(2):e12607.}

\leavevmode\hypertarget{ref-Nunes2020}{}%
\CSLLeftMargin{27. }
\CSLRightInline{Nunes S, Oliveira L, Siqueira J, Morton DC, Souza CM
(2020) {Unmasking secondary vegetation dynamics in the Brazilian
Amazon}. \emph{Environmental Research Letters} 15(3):034057.}

\leavevmode\hypertarget{ref-Smith2020}{}%
\CSLLeftMargin{28. }
\CSLRightInline{Smith CC, et al. (2020) {Secondary forests offset less
than 10{\%} of deforestation‐mediated carbon emissions in the Brazilian
Amazon}. \emph{Global Change Biology} 6:gcb.15352.}

\leavevmode\hypertarget{ref-Schwartz2020}{}%
\CSLLeftMargin{29. }
\CSLRightInline{Schwartz NB, Aide TM, Graesser J, Grau HR, Uriarte M
(2020) {Reversals of Reforestation Across Latin America Limit Climate
Mitigation Potential of Tropical Forests}. \emph{Frontiers in Forests
and Global Change} 3(July):1--10.}

\leavevmode\hypertarget{ref-Aguiar2016}{}%
\CSLLeftMargin{30. }
\CSLRightInline{Aguiar APD, et al. (2016) {Land use change emission
scenarios: Anticipating a forest transition process in the Brazilian
Amazon}. \emph{Global Change Biology} 22(5):1821--1840.}

\leavevmode\hypertarget{ref-Vancutsem2021}{}%
\CSLLeftMargin{31. }
\CSLRightInline{Vancutsem C, et al. (2021) {Long-term (1990--2019)
monitoring of forest cover changes in the humid tropics}. \emph{Science
Advances} 7(10):1--22.}

\leavevmode\hypertarget{ref-Campbell2008}{}%
\CSLLeftMargin{32. }
\CSLRightInline{Campbell JE, Lobell DB, Genova RC, Field CB (2008) {The
global potential of bioenergy on abandoned agriculture lands}.
\emph{Environmental Science and Technology} 42(15):5791--5794.}

\leavevmode\hypertarget{ref-Norden2015}{}%
\CSLLeftMargin{33. }
\CSLRightInline{Norden N, et al. (2015) {Successional dynamics in
Neotropical forests are as uncertain as they are predictable}.
\emph{Proceedings of the National Academy of Sciences}
112(26):8013--8018.}

\leavevmode\hypertarget{ref-Crouzeilles2016}{}%
\CSLLeftMargin{34. }
\CSLRightInline{Crouzeilles R, et al. (2016) {A global meta-analysis on
the ecological drivers of forest restoration success}. \emph{Nature
Communications} 7(1):11666.}

\leavevmode\hypertarget{ref-Arroyo-Rodriguez2017}{}%
\CSLLeftMargin{35. }
\CSLRightInline{Arroyo-Rodríguez V, et al. (2017) {Multiple successional
pathways in human-modified tropical landscapes: new insights from forest
succession, forest fragmentation and landscape ecology research}.
\emph{Biological Reviews} 92(1):326--340.}

\leavevmode\hypertarget{ref-Meli2017}{}%
\CSLLeftMargin{36. }
\CSLRightInline{Meli P, et al. (2017) {A global review of past land use,
climate, and active vs. passive restoration effects on forest recovery}.
\emph{PLOS ONE} 12(2):e0171368.}

\leavevmode\hypertarget{ref-Rydgren2020}{}%
\CSLLeftMargin{37. }
\CSLRightInline{Rydgren K, et al. (2020) {Assessing restoration success
by predicting time to recovery---But by which metric?} \emph{Journal of
Applied Ecology} 57(2):390--401.}

\leavevmode\hypertarget{ref-Jones2009}{}%
\CSLLeftMargin{38. }
\CSLRightInline{Jones HP, Schmitz OJ (2009) {Rapid recovery of damaged
ecosystems}. \emph{PLoS ONE} 4(5).
doi:\href{https://doi.org/10.1371/journal.pone.0005653}{10.1371/journal.pone.0005653}.}

\leavevmode\hypertarget{ref-Kampf2016}{}%
\CSLLeftMargin{39. }
\CSLRightInline{Kämpf I, Mathar W, Kuzmin I, Hölzel N, Kiehl K (2016)
{Post-Soviet recovery of grassland vegetation on abandoned fields in the
forest steppe zone of Western Siberia}. \emph{Biodiversity and
Conservation} 25(12):2563--2580.}

\leavevmode\hypertarget{ref-Kamp2011}{}%
\CSLLeftMargin{40. }
\CSLRightInline{Kamp J, Urazaliev R, Donald PF, Hölzel N (2011)
{Post-Soviet agricultural change predicts future declines after recent
recovery in Eurasian steppe bird populations}. \emph{Biological
Conservation} 144(11):2607--2614.}

\leavevmode\hypertarget{ref-Martin2013}{}%
\CSLLeftMargin{41. }
\CSLRightInline{Martin PA, Newton AC, Bullock JM (2013) {Carbon pools
recover more quickly than plant biodiversity in tropical secondary
forests}. \emph{Proceedings of the Royal Society B: Biological Sciences}
280(1773).
doi:\href{https://doi.org/10.1098/rspb.2013.2236}{10.1098/rspb.2013.2236}.}

\leavevmode\hypertarget{ref-Yang2019}{}%
\CSLLeftMargin{42. }
\CSLRightInline{Yang Y, Tilman D, Furey G, Lehman C (2019) {Soil carbon
sequestration accelerated by restoration of grassland biodiversity}.
\emph{Nature Communications} 10(718):1--7.}

\leavevmode\hypertarget{ref-Crouzeilles2019}{}%
\CSLLeftMargin{43. }
\CSLRightInline{Crouzeilles R, et al. (2019) {A new approach to map
landscape variation in forest restoration success in tropical and
temperate forest biomes}. \emph{Journal of Applied Ecology}
(August):1365--2664.13501.}

\leavevmode\hypertarget{ref-Parkhurst2021}{}%
\CSLLeftMargin{44. }
\CSLRightInline{Parkhurst T, Prober SM, Hobbs RJ, Standish RJ (2021)
{Global meta-analysis reveals incomplete recovery of soil conditions and
invertebrate assemblages after ecological restoration in agricultural
landscapes}. \emph{Journal of Applied Ecology} (April 2020):1--15.}

\leavevmode\hypertarget{ref-Crouzeilles2017}{}%
\CSLLeftMargin{45. }
\CSLRightInline{Crouzeilles R, et al. (2017) {Ecological restoration
success is higher for natural regeneration than for active restoration
in tropical forests}. \emph{Science Advances} 3(11):1--8.}

\leavevmode\hypertarget{ref-Queiroz2014}{}%
\CSLLeftMargin{46. }
\CSLRightInline{Queiroz C, Beilin R, Folke C, Lindborg R (2014)
{Farmland abandonment: Threat or opportunity for biodiversity
conservation? A global review}. \emph{Frontiers in Ecology and the
Environment} 12(5):288--296.}

\leavevmode\hypertarget{ref-Sirami2008}{}%
\CSLLeftMargin{47. }
\CSLRightInline{Sirami C, Brotons L, Burfield I, Fonderflick J, Martin
JL (2008) {Is land abandonment having an impact on biodiversity? A
meta-analytical approach to bird distribution changes in the
north-western Mediterranean}. \emph{Biological Conservation}
141(2):450--459.}

\leavevmode\hypertarget{ref-Cook-Patton2021}{}%
\CSLLeftMargin{48. }
\CSLRightInline{Cook-Patton SC, Shoch D, Ellis PW (2021) {Dynamic global
monitoring needed to use restoration of forest cover as a climate
solution}. \emph{Nature Climate Change} 11(5):366--368.}

\leavevmode\hypertarget{ref-Prishchepov2021a}{}%
\CSLLeftMargin{49. }
\CSLRightInline{Prishchepov AV, Ponkina EV, Sun Z, Bavorova M,
Yekimovskaja OA (2021) {Revealing the intentions of farmers to
recultivate abandoned farmland: A case study of the Buryat Republic in
Russia}. \emph{Land Use Policy} 107(July 2020):105513.}

\leavevmode\hypertarget{ref-Hua2016}{}%
\CSLLeftMargin{50. }
\CSLRightInline{Hua F, et al. (2016) {Opportunities for biodiversity
gains under the world's largest reforestation programme}. \emph{Nature
Communications} 7:12717.}

\leavevmode\hypertarget{ref-FAO2016}{}%
\CSLLeftMargin{51. }
\CSLRightInline{Food and Agriculture Organization of the United Nations
(FAO) (2016) {FAOSTAT Statistical Database: Methods {\&} Standards}.
Available at:
\url{http://www.fao.org/ag/agn/nutrition/Indicatorsfiles/Agriculture.pdf\%20http://www.fao.org/faostat/en/\%7B/\#\%7Ddefinitions}.}

\leavevmode\hypertarget{ref-RStudio}{}%
\CSLLeftMargin{52. }
\CSLRightInline{RStudio Team (2020) {RStudio: Integrated Development
Environment for R}. Available at: \url{http://www.rstudio.com/}.}

\leavevmode\hypertarget{ref-R-raster}{}%
\CSLLeftMargin{53. }
\CSLRightInline{Hijmans RJ (2021) \emph{Raster: Geographic data analysis
and modeling} Available at: \url{https://rspatial.org/raster}.}

\leavevmode\hypertarget{ref-R-data.table}{}%
\CSLLeftMargin{54. }
\CSLRightInline{Dowle M, Srinivasan A (2021) \emph{Data.table: Extension
of `data.frame`}.}

\leavevmode\hypertarget{ref-R-tidyverse}{}%
\CSLLeftMargin{55. }
\CSLRightInline{Wickham H (2021) \emph{Tidyverse: Easily install and
load the tidyverse} Available at:
\url{https://CRAN.R-project.org/package=tidyverse}.}

\end{CSLReferences}

This PNAS journal template is provided to help you write your work in
the correct journal format. Instructions for use are provided below.

Note: please start your introduction without including the word
``Introduction'' as a section heading (except for math articles in the
Physical Sciences section); this heading is implied in the first
paragraphs.

\hypertarget{guide-to-using-this-template}{%
\section*{Guide to using this
template}\label{guide-to-using-this-template}}
\addcontentsline{toc}{section}{Guide to using this template}

Please note that whilst this template provides a preview of the typeset
manuscript for submission, to help in this preparation, it will not
necessarily be the final publication layout. For more detailed
information please see the
\href{http://www.pnas.org/site/authors/format.xhtml}{PNAS Information
for Authors}.

\hypertarget{author-affiliations}{%
\subsection*{Author Affiliations}\label{author-affiliations}}
\addcontentsline{toc}{subsection}{Author Affiliations}

Include department, institution, and complete address, with the
ZIP/postal code, for each author. Use lower case letters to match
authors with institutions, as shown in the example. Authors with an
ORCID ID may supply this information at submission.

\hypertarget{submitting-manuscripts}{%
\subsection*{Submitting Manuscripts}\label{submitting-manuscripts}}
\addcontentsline{toc}{subsection}{Submitting Manuscripts}

All authors must submit their articles at
\href{http://www.pnascentral.org/cgi-bin/main.plex}{PNAScentral}. If you
are using Overleaf to write your article, you can use the ``Submit to
PNAS'' option in the top bar of the editor window.

\hypertarget{format}{%
\subsection*{Format}\label{format}}
\addcontentsline{toc}{subsection}{Format}

Many authors find it useful to organize their manuscripts with the
following order of sections; Title, Author Affiliation, Keywords,
Abstract, Significance Statement, Results, Discussion, Materials and
methods, Acknowledgments, and References. Other orders and headings are
permitted.

\hypertarget{manuscript-length}{%
\subsection*{Manuscript Length}\label{manuscript-length}}
\addcontentsline{toc}{subsection}{Manuscript Length}

PNAS generally uses a two-column format averaging 67 characters,
including spaces, per line. The maximum length of a Direct Submission
research article is six pages and a PNAS PLUS research article is ten
pages including all text, spaces, and the number of characters displaced
by figures, tables, and equations. When submitting tables, figures,
and/or equations in addition to text, keep the text for your manuscript
under 39,000 characters (including spaces) for Direct Submissions and
72,000 characters (including spaces) for PNAS PLUS.

\hypertarget{references}{%
\subsection*{References}\label{references}}
\addcontentsline{toc}{subsection}{References}

References should be cited in numerical order as they appear in text;
this will be done automatically via bibtex, e.g.
(\textbf{belkin2002using?}) and (\textbf{berard1994embedding?},
\textbf{coifman2005geometric?}). All references, including for the SI,
should be included in the main manuscript file. References appearing in
both sections should not be duplicated. SI references included in tables
should be included with the main reference section.

\hypertarget{data-archival}{%
\subsection*{Data Archival}\label{data-archival}}
\addcontentsline{toc}{subsection}{Data Archival}

PNAS must be able to archive the data essential to a published article.
Where such archiving is not possible, deposition of data in public
databases, such as GenBank, ArrayExpress, Protein Data Bank, Unidata,
and others outlined in the Information for Authors, is acceptable.

\hypertarget{language-editing-services}{%
\subsection*{Language-Editing
Services}\label{language-editing-services}}
\addcontentsline{toc}{subsection}{Language-Editing Services}

Prior to submission, authors who believe their manuscripts would benefit
from professional editing are encouraged to use a language-editing
service (see list at www.pnas.org/site/authors/language-editing.xhtml).
PNAS does not take responsibility for or endorse these services, and
their use has no bearing on acceptance of a manuscript for publication.

\begin{figure}
\centering
\includegraphics{frog.png}
\caption{Placeholder image of a frog with a long example caption to show
justification setting.{}}
\end{figure}

\hypertarget{sec:figures}{%
\subsection*{Digital Figures}\label{sec:figures}}
\addcontentsline{toc}{subsection}{Digital Figures}

Only TIFF, EPS, and high-resolution PDF for Mac or PC are allowed for
figures that will appear in the main text, and images must be final
size. Authors may submit U3D or PRC files for 3D images; these must be
accompanied by 2D representations in TIFF, EPS, or high-resolution PDF
format. Color images must be in RGB (red, green, blue) mode. Include the
font files for any text.

Figures and Tables should be labelled and referenced in the standard way
using the \texttt{\textbackslash{}label\{\}} and
\texttt{\textbackslash{}ref\{\}} commands.

Figure \[fig:frog\] shows an example of how to insert a column-wide
figure. To insert a figure wider than one column, please use the
\texttt{\textbackslash{}begin\{figure*\}...\textbackslash{}end\{figure*\}}
environment. Figures wider than one column should be sized to 11.4 cm or
17.8 cm wide.

\hypertarget{single-column-equations}{%
\subsection*{Single column equations}\label{single-column-equations}}
\addcontentsline{toc}{subsection}{Single column equations}

Authors may use 1- or 2-column equations in their article, according to
their preference.

To allow an equation to span both columns, options are to use the
\texttt{\textbackslash{}begin\{figure*\}...\textbackslash{}end\{figure*\}}
environment mentioned above for figures, or to use the
\texttt{\textbackslash{}begin\{widetext\}...\textbackslash{}end\{widetext\}}
environment as shown in equation \[eqn:example\] below.

Please note that this option may run into problems with floats and
footnotes, as mentioned in the \href{http://texdoc.net/pkg/cuted}{cuted
package documentation}. In the case of problems with footnotes, it may
be possible to correct the situation using commands
\texttt{\textbackslash{}footnotemark} and
\texttt{\textbackslash{}footnotetext}.

\[\begin{aligned}
(x+y)^3&=(x+y)(x+y)^2\\
       &=(x+y)(x^2+2xy+y^2) \label{eqn:example} \\
       &=x^3+3x^2y+3xy^3+x^3. 
\end{aligned}\]

\showmatmethods
\showacknow
\pnasbreak



% Bibliography
% \bibliography{pnas-sample}

\end{document}
